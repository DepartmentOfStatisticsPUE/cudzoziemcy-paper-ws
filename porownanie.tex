%%%%%%%%%%%%%%%%%%%%%%%%%%%%%%%%%%%%%%%%%%%%%%%%%%%%%%%%%%%%%%%%%%%%%%%%%%%%%%%%%%%%%%%%%%%%%
%DIF LATEXDIFF DIFFERENCE FILE
%DIF DEL oryg/main.tex      Sat Jun 29 10:45:24 2019
%DIF ADD revised/main.tex   Sat Jun 29 10:44:48 2019
\documentclass[12pt,a4paper]{article}
%%%%%%%%%%%%%%%%%%%%%%%%%%%%%%%%%%%%%%%%%%%%%%%%%%%%%%%%%%%%%%%%%%%%%%%%%%%%%%%%%%%%%%%%%%%%%
\input{settings.tex}
%%%%%%%%%%%%%%%%%%%%%%%%%%%%%%%%%%%%%%%%%%%%%%%%%%%%%%%%%%%%%%%%%%%%%%%%%%%%%%%%%%%%%%%%%%%%%   
\title{Estymacja liczby cudzoziemców w~Polsce w~latach 2015 i~2016 z~wykorzystaniem rejestrów administracyjnych i~metody capture-recapture}

\author{\large{Maciej Beręsewicz}, \large{Grzegorz Gudaszewski}, \large{Marcin Szymkowiak}}

\date{} 
\makeatletter
\def\thesection{\arabic{section}.}
\def\thesubsection{\thesection\arabic{subsection}.}
\def\thesubsubsection{\thesubsection\arabic{subsubsection}.}
\makeatother
%DIF PREAMBLE EXTENSION ADDED BY LATEXDIFF
%DIF UNDERLINE PREAMBLE %DIF PREAMBLE
\RequirePackage[normalem]{ulem} %DIF PREAMBLE
\RequirePackage{color}\definecolor{RED}{rgb}{1,0,0}\definecolor{BLUE}{rgb}{0,0,1} %DIF PREAMBLE
\providecommand{\DIFadd}[1]{{\protect\color{blue}\uwave{#1}}} %DIF PREAMBLE
\providecommand{\DIFdel}[1]{{\protect\color{red}\sout{#1}}}                      %DIF PREAMBLE
%DIF SAFE PREAMBLE %DIF PREAMBLE
\providecommand{\DIFaddbegin}{} %DIF PREAMBLE
\providecommand{\DIFaddend}{} %DIF PREAMBLE
\providecommand{\DIFdelbegin}{} %DIF PREAMBLE
\providecommand{\DIFdelend}{} %DIF PREAMBLE
\providecommand{\DIFmodbegin}{} %DIF PREAMBLE
\providecommand{\DIFmodend}{} %DIF PREAMBLE
%DIF FLOATSAFE PREAMBLE %DIF PREAMBLE
\providecommand{\DIFaddFL}[1]{\DIFadd{#1}} %DIF PREAMBLE
\providecommand{\DIFdelFL}[1]{\DIFdel{#1}} %DIF PREAMBLE
\providecommand{\DIFaddbeginFL}{} %DIF PREAMBLE
\providecommand{\DIFaddendFL}{} %DIF PREAMBLE
\providecommand{\DIFdelbeginFL}{} %DIF PREAMBLE
\providecommand{\DIFdelendFL}{} %DIF PREAMBLE
%DIF END PREAMBLE EXTENSION ADDED BY LATEXDIFF

\begin{document}

\maketitle

\textbf{Streszczenie.} \DIFdelbegin %DIFDELCMD < \textit{%%%
\DIFdel{Istnieje duże zapotrzebowanie na określenie rzeczywistych rozmiarów }\DIFdelend \DIFaddbegin \DIFadd{W~pracy zaproponowano metody szacunku }\DIFaddend wielkości \DIFdelbegin \DIFdel{zasobu }\DIFdelend \DIFaddbegin \DIFadd{populacji }\DIFaddend cudzoziemców przebywających \DIFdelbegin \DIFdel{w~Polsce (z~}\DIFdelend \DIFaddbegin \DIFadd{w~Polsce na koniec 2015 i~2016 roku, ze szczególnym }\DIFaddend uwzględnieniem \DIFdelbegin \DIFdel{nierejestrowanej/nieudokumentowanej imigracji), w~szczególności w~obszarze imigracji pracowniczych. Dane dotyczące zasobów imigracyjnych są bowiem ważnym elementem prowadzenia polityki spójności, zwłaszcza, jeśli weźmie }\DIFdelend \DIFaddbegin \DIFadd{cudzoziemców pracujących na terytorium Polski. W tym celu wykorzystano administracyjne źródła danych oraz techniki bazujące na metodzie }\textit{\DIFadd{capture-recapture}}\DIFadd{, ze szczególnym uwzględnieniem modeli log-liniowych. Szacuje }\DIFaddend się\DIFdelbegin \DIFdel{pod uwagę fakt}\DIFdelend , \DIFdelbegin \DIFdel{iż imigracja cudzoziemców nie jest zjawiskiem dotykającym w~równym stopniu cały obszar kraju. Z~tego punktu widzenia niezwykle ważne jest wypracowanie odpowiedniej metody, która umożliwiłaby estymację liczby cudzoziemców w~Polsce dla dodatkowo określonych przekrojów (grupy wieku, płeć, klasa miejsca zamieszkania itd.) z~akceptowalnym błędem szacunku. Ze względu na fakt, }\DIFdelend że \DIFdelbegin \DIFdel{zbiorowość }\DIFdelend \DIFaddbegin \DIFadd{w~2015 i~2016 roku na terenie Polski mogło przebywać odpowiednio około 500 tys. (95\% przedział ufności 369--724 tys.) oraz około 744 tys. (601--943 tys.) }\DIFaddend cudzoziemców\DIFdelbegin \DIFdel{w~Polsce stanowi populację trudną do zbadania, do tej pory nie było kompleksowych opracowań stanowiących próbę rozwiązania tego problemu}\footnote{\DIFdel{Termin ,,populacji trudnej do zbadania'' (ang. }\textit{\DIFdel{hard-to-survey population}}%DIFAUXCMD
\DIFdel{) zostanie szerzej omówiony w~dalszej części artykułu.}}%DIFAUXCMD
\addtocounter{footnote}{-1}%DIFAUXCMD
\DIFdel{.}%DIFDELCMD < \MBLOCKRIGHTBRACE 
%DIFDELCMD < %%%
\DIFdelend \DIFaddbegin \DIFadd{. Jest to pierwsza tego typu kompleksowa analiza dotycząca próby estymacji liczby cudzoziemców w~Polsce, która wpisuje się w~nurt badań nad populacjami trudnymi do zbadania. Należy jednak mieć na uwadze konieczność spełnienia założeń tej metody, co również stanowiło obszar rozważań autorów w~niniejszym artykule.
}\DIFaddend 

\DIFdelbegin \textit{\DIFdel{Głównym celem artykułu jest przedstawienie wybranych wyników uzyskanych w~ramach pracy badawczej pt. }\textbf{\DIFdel{Cudzoziemcy na krajowym rynku pracy w~ujęciu regionalnym}} %DIFAUXCMD
\DIFdel{realizowanej we współpracy przez pracowników Głównego Urzędu Statystycznego w~Warszawie, Urzędu Statystycznego w~Poznaniu oraz Uniwersytetu Ekonomicznego w~Poznaniu. W~pracy zaproponowano metody szacunku zasobów cudzoziemców przebywających w~Polsce, ze szczególnym uwzględnieniem cudzoziemców pracujących na terytorium Polski, w~oparciu o~administracyjne i~statystyczne źródła danych oraz przy wykorzystaniu technik bazujących na metodzie }\textit{\DIFdel{capture-recapture}} %DIFAUXCMD
\DIFdel{z~wykorzystaniem modeli log-liniowych. Jest to pierwsza tego typu kompleksowa analiza dotycząca próby estymacji liczby cudzoziemców w~Polsce, która wpisuje się w~nurt badań nad populacjami trudnymi do zbadania.}}
%DIFAUXCMD
%DIFDELCMD < 

%DIFDELCMD < %%%
\DIFdelend \textbf{Słowa kluczowe:} estymacja liczby cudzoziemców, populacja trudna do zbadania, metody capture-recapture, analiza log-liniowa, rejestry administracyjne

\DIFaddbegin \vspace{0.3cm}

\textbf{\DIFadd{Estimation the number of foreigners in Poland in 2015 and 2016 using administrative registers and the capture-recapture method}}

\vspace{0.3cm}

\textbf{\DIFadd{Summmary.}} \DIFadd{The article describes methods of estimating the size of the foreigner population in Poland at the turn of 2015/2016, with special emphasis on foreigners engaged in paid employment.  To achieve this goal the authors used administrative data sources and techniques of the capture-recapture method as well as log-linear models. The number of foreigners staying in Poland in 2015 and 2016 was estimated to be at about 500,000 (95\% CI: 369--724,000) and about 744,000 (601-943,000), respectively.  The study is the first comprehensive attempt to estimate the number of foreigners in Poland, which is an example of research into hard-to-survey populations. However, one should be aware of restrictive assumptions of the capture-recapte methods which will be also discussed in depth in the paper.
}

\textbf{\DIFadd{Keywords:}} \DIFadd{estimation the number of foreigners in Poland, hard-to-survey population,  capture-recapture methods, log-linear analysis, administrative registers
}

\DIFaddend \textbf{JEL:} C81, J61, J68, F22
%%%%%%%%%%%%%%%%%%%%%%%%%%%%%%%%%%%%%%%%%%%%%%%%%%%%%%%%%%%%%%%%%%%%%%%%%%%%%%%%%%%%%%%%%%%%%
\DIFaddbegin 

\DIFaddend \begin{center}
\DIFdelbegin \section*{\DIFdel{WSTĘP}}
%DIFAUXCMD
\DIFdelend \DIFaddbegin \DIFadd{WSTĘP
}\DIFaddend \end{center}

Coraz częściej, zarówno na szczeblu rządowym, samorządowym jak i~lokalnym, podnoszona jest kwestia konieczności dysponowania rzeczywistą skalą \DIFdelbegin \DIFdel{zasobu }\DIFdelend \DIFaddbegin \DIFadd{liczby }\DIFaddend cudzoziemców przebywających w~Polsce stale i~czasowo oraz~podejmujących pracę. Szczególnie istotna dla realizowania polityk ludnościowych, migracyjnych i~gospodarczych jest informacja o~cechach demograficzno-społecznych i~ekonomicznych cudzoziemców. Kolejna ważna kwestia dotyczy skali imigracji nierejestrowanej, tj. pozostającej poza ewidencją. Obecnie nie ma w~Polsce miarodajnego i~bezpośredniego źródła danych, które dostarczałoby wiarygodnych informacji w~tym zakresie. Zwrócić należy również uwagę na fakt, iż imigracja cudzoziemców nie jest zjawiskiem dotykającym obszaru całego kraju w~równym stopniu i~podlega zróżnicowaniu przestrzennemu, zwłaszcza w~kontekście regionalnych rynków pracy. 

Posiadanie informacji na temat liczby cudzoziemców, uwzględniając w~tym nierejestrowanych imigrantów, stanowi dla służb statystyki publicznej w~Polsce istotne wyzwanie metodologiczne. Po pierwsze, rejestry administracyjne dostarczają informacji o~populacji \textit{de \DIFdelbegin \DIFdel{jure}\DIFdelend \DIFaddbegin \DIFadd{iure}\DIFaddend } (zarejestrowanej), podczas gdy statystyka zainteresowana jest populacją \textit{de facto} (zarejestrowanej i~niezarejestrowanej). Po drugie, cudzoziemcy stanowią populację trudną do zbadania przy wykorzystaniu tradycyjnych metod statystycznych. Populacja taka charakteryzuje się bowiem brakiem dostępnego (wyczerpującego) operatu losowania oraz trudnością w~pozyskaniu informacji od jednostek do niej należących. O~ile rozpoznanie problemów występujących w~populacjach trudnych do zbadania jest możliwe na gruncie badań statystycznych (można zastosować przykładowo dobór jednostek do badania w~oparciu o~metodę kuli śnieżnej i~jej rozszerzenie -- metodę RDS\footnote{ang. \textit{Respondent Driven Sampling} -- metoda doboru jednostek do próby sterowana przez respondentów. Jest to zmodyfikowana wersja metody kuli śnieżnej, w~której stosuje się podwójny system zachęt polegający na wynagrodzeniu respondenta za wzięcie udziału w~badaniu jak i~zwerbowaniu kolejnych osób, które biorą w~nim udział. W~metodzie RDS wykorzystuje się informacje na temat sieci powiązań osób należących do danej zbiorowości.}), o~tyle proces estymacji wielkości takiej zbiorowości jest z~metodologicznego punktu widzenia poważnym wyzwaniem badawczym. W~literaturze przedmiotu istnieją jednak odpowiednie metody statystyczne, które umożliwiają estymację wielkości populacji trudnych do zbadania bazujące na technikach capture-recapture \citep[por.][]{bohning-capture-recapture-2017}. Zaliczyć tutaj można rozwiązania, w~których wykorzystuje się jedno \citep[por.][]{van2003point, godwin2017estimation} albo co najmniej dwa źródła danych \citep[por.][]{van2012people, zhang2008developing}. Skuteczne wykorzystanie tych technik w~praktyce w~dużej mierze zależy od dostępności danych statystycznych i~jest uwarunkowane koniecznością spełnienia odpowiednich założeń leżących u~podstaw poszczególnych metod.

W~artykule podjęto próbę oszacowania wielkości \DIFdelbegin \DIFdel{zasobów }\DIFdelend \DIFaddbegin \DIFadd{populacji }\DIFaddend cudzoziemców przebywających w~Polsce w~końcu 2015 i~2016 roku na \DIFdelbegin \DIFdel{poziomie }\DIFdelend \DIFaddbegin \DIFadd{terenie }\DIFaddend kraju i~według \DIFdelbegin \DIFdel{wybranych cech demograficznych}\DIFdelend \DIFaddbegin \DIFadd{kraju obywatelstwa}\DIFaddend \footnote{W~pracy badawczej, na podstawie której powstał niniejszy artykuł, rozważany był również poziom województw i~podregionów. Opracowano także podstawową charakterystykę cudzoziemców uwzględniającą wybrane cechy demograficzno-społeczne, obywatelstwo czy status na rynku pracy na podstawie danych z~Narodowego Spisu Powszechnego Ludności i~Mieszkań 2011, Badania Aktywności Ekonomicznej Ludności oraz zezwoleń i~oświadczeń Ministerstwa Rodziny, Pracy i~Polityki Społecznej, które nie będą jednak omawiane w~niniejszym artykule.}. \DIFaddbegin \DIFadd{Przyjęto przy tym następującą definicję cudzoziemca -- osoba nieposiadająca obywatelstwa polskiego lub bezpaństwowiec (podstawa prawna – ustawa z dnia 12 grudnia 2013 r. o cudzoziemcach, Dz.U. z 2016 r. poz. 1990 z późn. zm.).  }\DIFaddend W~tym celu posłużono się odpowiednio zbudowanym modelem log-liniowym z~szeregiem zmiennych pomocniczych. Przyjęto przy tym definicję cudzoziemca jako osoby nieposiadającej obywatelstwa polskiego lub bezpaństwowca\footnote{Podstawa prawna – ustawa z~dnia 12 grudnia 2013 r. o~cudzoziemcach, Dz.U. z~2016 r. poz. 1990 z~późn. zm.}. 

W~kolejnych częściach artykułu przedstawiono przegląd literatury w~zakresie estymacji populacji trudnych do zbadania. Opisano przy tym ideę metody capture-recapture oraz jej zastosowań w~kontekście szacowania liczby cudzoziemców. Następnie omówiono wykorzystane źródła danych, a~także opisano wybrane aspekty modeli log-liniowych, które stanowiły podstawową metodę estymacji wykorzystaną w~procesie szacowania liczby cudzoziemców w~Polsce w~latach 2015--2016. Należy zaznaczyć, że wybór okresu oraz źródeł danych był podyktowany ich aktualną dostępnością dla statystyki publicznej w~ramach \DIFdelbegin \DIFdel{PBSSP}\footnote{\DIFdel{Program badań statystycznych statystyki publicznej.}} %DIFAUXCMD
\addtocounter{footnote}{-1}%DIFAUXCMD
\DIFdelend \DIFaddbegin \DIFadd{Program badań statystycznych statystyki publicznej (PBSSP) }\DIFaddend i~niemożliwe było pozyskanie innych danych jednostkowych. \DIFdelbegin \DIFdel{Dlatego do prezentowanych wyników należy podchodzić z~ostrożnością. }\DIFdelend \DIFaddbegin \DIFadd{Udostępnione dane były aktualne na 31.12.2016~r. Należy podkreślić, że nie oznacza to, że w tym dniu wszyscy badani cudzoziemcy przebywali na terenie Polski, podobnie jak w~rejestrze PESEL, który nie gwarantuje, że na terenie Polski w danym dniu przebywa określona liczba obywateli polskich. }\DIFaddend W~dalszej części artykułu przedstawiono także w~jaki sposób \DIFdelbegin \DIFdel{spełniono }\DIFdelend \DIFaddbegin \DIFadd{podjęto próbę spełnienia }\DIFaddend założenia metody \textit{capture-recapture}, wyniki estymacji z~uwzględnieniem wybranych zmiennych demograficznych i~porównanie do liczebności z~rejestrów administracyjnych. Całość artykułu stanowi podsumowanie, w~którym sformułowano dalsze kroki badawcze. 

\vspace{0.5cm}

%%%%%%%%%%%%%%%%%%%%%%%%%%%%%%%%%%%%%%%%%%%%%%%%%%%%%%%%%%%%%%%%%%%%%%%%%%%%%%%%%%%%%%%%%%%%%
\begin{center}
\DIFdelbegin \section*{\DIFdel{ESTYMACJA LICZEBNOŚCI POPULACJI TRUDNEJ DO ZBADANIA -- PRZEGLĄD LITERATURY}}
%DIFAUXCMD
\DIFdelend \DIFaddbegin \DIFadd{ESTYMACJA LICZEBNOŚCI POPULACJI TRUDNEJ DO ZBADANIA -- PRZEGLĄD LITERATURY
}\DIFaddend \end{center}

\begin{center}
\DIFdelbegin \subsection*{\DIFdel{Populacje trudne do zbadania}}
%DIFAUXCMD
\DIFdelend \DIFaddbegin \textbf{\DIFadd{Populacje trudne do zbadania}}
\DIFaddend \end{center}

W~literaturze przedmiotu populacje trudne do zbadania można rozumieć na wiele różnych sposobów \citep{tourangeau2014hard}. W~gruncie rzeczy, ze względu na fakt, że w~wielu badaniach częściowych mamy do czynienia z~dużą frakcją odmów, terminem tym można byłoby określić każdą z~badanych populacji. Populacja trudna do zbadania ma jednak inne znaczenie i~odnosi się do zbiorowości, które przedstawiają szczególne wyzwania metodologiczne różnego rodzaju oraz sprawiają, że są trudniejsze do zbadania w~porównaniu z~innymi populacjami. Niektóre z~trudności związane mogą być z~tym, że są to populacje rzadkie, ukryte, z~jednostkami których trudno nawiązać kontakt czy ciężko współpracować. 

Mówiąc o~populacjach trudnych do zbadania należy rozróżnić populacje \citep{tourangeau2014hard}:
\begin{itemize}
\item z~których jednostki trudno wylosować do próby (ang. \textit{hard-to-sample}) -- w~przypadku populacji trudnych do zbadania bardzo rzadko zdarza się, aby istniał właściwy operat losowania, z~którego jednostki można byłoby wylosować do próby wykorzystując odpowiedni schemat jej pobierania. Z~tego względu, w~odniesieniu do takich populacji, stosuje się nielosowe dobory jednostek do próby, wśród których szczególną rolę odgrywają wspomniana już metoda kuli śnieżnej czy metoda doboru sterowana przez respondenta. Można również zastosować inne techniki doboru jednostek, zwłaszcza w~odniesieniu dla populacji rzadkich i~trudno uchwytnych, takie jak losowanie odwrotne, lokacyjne czy schematy linia-przecięcie oraz śledzenia łączy \citep{jedkub}. Istnieją jednak nawet i~w~takim przypadku trudności z~doborem jednostek do próby gdyż populacje takie mogą być mobilne bądź nieuchwytne\DIFdelbegin \DIFdel{, tj. niezwiązane z~żadnym miejscem na stałe}\DIFdelend . Przykładem tego typu populacji mogą być osoby bezdomne lub pracujący cudzoziemcy;
\item których jednostki trudno jest zidentyfikować (ang. \textit{hard-to-identify}) --  w~niektórych przypadkach, szczególnie w~odniesieniu do stygmatyzowanych grup społecznych, członkowie populacji mogą nie chcieć udostępnić swoich cech, co wiązać się może z~lękiem przed ujawnieniem nielegalnego lub wstydliwego statusu społecznego. W~takim przypadku utrudniona jest identyfikacja jednostek należących do takich populacji. Przykład tego typu populacji stanowić mogą narkomani, alkoholicy czy różne mniejszości (na przykład osoby LGBT, wyznawcy określonych religii czy ideologii);
\item których jednostki trudno znaleźć i~nawiązać z~nimi kontakt (ang. \textit{hard-to-find-and-contact}) -- trudność w~nawiązaniu kontaktu związana jest przede wszystkim z~mobilnością tego typu populacji. Przykładem tego typu populacji mogą być niezameldowani cudzoziemcy, członkowie koczowniczych kultur (Beduini z~południowo-zachodniej Azji czy Tuaregowie z~Afryki Północnej), mniejszości wędrowne (Romowie w~Europie), osoby bezdomne;  
\item której jednostki trudno namówić do wzięcia udziału w~badaniu (ang. \textit{hard-to-per\-su\-ade}) -- niechęć do wzięcia udziału w~badaniu związana może być z~drażliwością poruszanej tematyki bądź z~brakiem czasu. Przykładem tego typu populacji mogą być aktywni zawodowo, pracujący w~szarej czy czarnej strefie, cudzoziemcy;
\item której jednostki można zachęcić do wzięcia udziału w~badaniu, ale ciężko przeprowadzić wywiad (ang. \textit{hard-to-interview}) -- trudność w~przeprowadzeniu wywiadu może wynikać z~tego, że należy uzyskać na udział w~badaniu danej jednostki zgody przełożonego, opiekuna prawnego czy rodzica. Trudność ta może być również konsekwencją występowania niepełnosprawności czy bariery językowej, jeśli osoba ankietowana nie mówi w~języku, w~którym przygotowany został odpowiedni kwestionariusz. Wreszcie może być ona pochodną tego, że badanie należy przeprowadzić w~obszarze konfliktu zbrojnego. Przykład tego typu populacji stanowić mogą więźniowie, osoby niepełnosprawne psychicznie czy cudzoziemcy nie znający języka danego kraju. 
\end{itemize}

Jak pokazują powyższe rozważania trudność w~zbadaniu określonych populacji może być pochodną wielu czynników. Tak jest w~przypadku populacji cudzoziemców w~Polsce, której jednostki trudno wylosować do próby (brak pełnego operatu losowania oraz kompleksowych źródeł danych statystycznych, z~których można czerpać wiedzę na temat cudzoziemców), z~którą trudno nawiązać kontakt (mobilność cudzoziemców na rynku pracy oraz brak stałego miejsca zamieszkania) czy też przeszkodą może być bariera językowa. Czynniki te powodują również, że estymacja liczebności tego typu populacji, zwłaszcza z~uwzględnieniem dodatkowych przekrojów, jest niezwykle złożonym zadaniem. W~literaturze przedmiotu proponuje się jednak pewne rozwiązania, które stanowić mogą swego rodzaju remedium na problemy związane z~określeniem rzeczywistych rozmiarów populacji trudnych do zbadania. Należą one do grupy technik określanych wspólnym terminem \textit{capture-recapture}\footnote{W~artykule używać będziemy angielskiego terminu capture-recapture (CR) w~związku z~nie do końca jasnym tłumaczeniem tego podejścia na język polski. Bezpośrednie tłumaczenie mogłoby brzmieć jako 'metodę wielokrotnego połowu' przy czym to określenie nie oddaje istoty tego podejścia. W~szczególności w~odniesieniu do rejestrów administracyjnych, w~których nie dokonujemy losowań czy 'połowów' jednostek.}. W~dalszej części artykułu wskażemy na kilka praktycznych zastosowań technik statystycznych wchodzących w~skład metod typu \textit{capture-recapture} w~szacowaniu liczebności takich populacji, również z~uwzględnieniem populacji cudzoziemców.

\begin{center}
\DIFdelbegin \subsection*{\DIFdel{Szacowanie wielkości populacji trudnej do zbadania z~wykorzystaniem metod capture-recapture}}    
%DIFAUXCMD
\DIFdelend \DIFaddbegin \textbf{\DIFadd{Szacowanie wielkości populacji trudnej do zbadania z~wykorzystaniem metod capture-recapture}}    
\DIFaddend \end{center}

Metody \textit{capture-recapture}, które wykorzystane zostały na potrzeby oszacowania liczby cudzoziemców w~Polsce, wywodzą się z~nauk przyrodniczych. Pierwotnie użyto ich do oszacowania liczby ryb w~jeziorze \DIFaddbegin \DIFadd{\mbox{%DIFAUXCMD
\citep{goudie2007captures}}\hspace{0pt}%DIFAUXCMD
}\DIFaddend . Idea tego podejścia polega na tym, że w~typowym badaniu z~obszaru nauk przyrodniczych przeprowadzanym metodą \textit{capture-recapture} na analizowanym terytorium umieszcza się pułapki lub siatki w~celu wielokrotnego wyłapywania osobników danej populacji. W~pierwszej próbie złowiona jest pewna liczba osobników, które po oznakowaniu są wypuszczane na wolność. W~każdej kolejnej próbie zapisuje się i~znakuje każde nieoznaczone zwierzę, notuje się każde zwierzę, które zostało wcześniej oznakowane i~ponownie wypuszcza się je na wolność. Po zakończeniu badania uzyskuje się pełną historię złowień dla każdego osobnika. Badania tego typu określane są jako badania \textit{mark–recapture}, \textit{tag–recapture}, czy \textit{multiple-record system}.  

W~najprostszej wersji metoda \textit{capture-recapture} składa się z~dwóch prób lub źródeł\footnote{Możliwe jest również zastosowanie metody w~przypadku jednego źródła o czym mowa później.}: pierwsza to próba zawierająca osobniki złowione za pierwszym razem i~druga zawierająca zwierzęta złowione za drugim razem. Ten szczególny przypadek złożony z~dwóch prób w~kontekście szacowania błędu niedostatecznego pokrycia określany jest jako system podwójny (ang. \textit{dual system}) lub system podwójnego zapisu (ang. \textit{dual-system record}). Od~wielu lat metodę wielokrotnych złowień stosuje się do szacowania parametrów demograficznych w~populacjach zwierzęcych. Biolodzy już dawno zauważyli, że nie jest konieczne, ani nawet możliwe, zliczenie wszystkich zwierząt w~celu dokładnego oszacowania wielkości populacji. Informacja na temat liczby ponownych złowień (lub proporcji ponownych złowień) uzyskiwana poprzez znakowanie odgrywa tu istotną rolę ponieważ można ją wykorzystać do oszacowania liczby osobników nie ujętych w~próbach przyjmując odpowiednie założenia.  
W~najprostszym ujęciu można założyć, że w~przypadku gdy liczba ponownie złowionych osobników w~kolejnych próbach jest niewielka, rozmiar populacji jest większy niż liczba unikatowych osobników, jakie zostały złowione. Natomiast jeśli wskaźnik ponownych złowień jest stosunkowo wysoki, można przypuszczać, że złowiona została większość zwierząt z~danej populacji. Pomysł zastosowania techniki złożonej z~dwóch prób można odnaleźć w~pracach Pierre’a Simona Laplace’\DIFdelbegin \DIFdel{s }\DIFdelend \DIFaddbegin \DIFadd{a }\DIFaddend z~1786 roku, który wykorzystał ją do szacowania liczby ludności Francji w~1802 roku, a~nawet wcześniej, w~pracach Johna Graunta, który zastosował tę technikę do szacowania skutków zarazy wśród ludności Anglii około roku 1600. W~dziedzinie ekologii technika ta najwcześniej użyta została w~badaniach Petersena i~Dahla dotyczących populacji ryb odpowiednio w~roku 1896 i~1907 oraz w~przeprowadzonym przez Lincolna badaniu powrotów zaobrączkowanych ptaków wodnych z~roku 1930. Modele oparte na dwóch próbach zostały rozszerzone na przypadki zawierające większą liczbę prób przez Schnabela w~roku 1938. Stąd też metoda wielokrotnych złowień nazywana jest również spisem Schnabela. Bardziej zaawansowana teoria statystyczna i~procedury wnioskowania pojawiły się po publikacji prac Darrocha, który opracował zagadnienie od strony matematycznej \citep[Rozdział 1]{bohning-bunge-hijden-2018}.

Założenia stosowane w~odniesieniu do populacji zwierzęcych klasyfikuje się generalnie jako modele zamknięte i~otwarte. W~przypadku zamkniętym zakłada się, że wielkość populacji, która jest przedmiotem badania, jest stała w~czasie prowadzonego badania. Założenie to jest zwykle zachowane w~przypadku danych zbieranych na przestrzeni stosunkowo krótkiego czasu poza okresem godowym. W~modelu otwartym, dopuszcza się przyrosty (narodziny lub imigracja) lub ubytki (śmierć lub emigracja) w~populacji. Założenie otwartej populacji jest zwykle wykorzystywane w~długoterminowych badaniach zwierząt lub ptaków wędrownych. Poza wielkością populacji w~momencie poszczególnych prób, badane parametry obejmują również wskaźnik przeżywalności oraz liczbę narodzin pomiędzy próbami. W~dalszej części uwaga skupiona zostanie na modelach zamkniętych w~odniesieniu do populacji ludzi. 

Warto również zaznaczyć, że współcześnie pojęcie \textit{capture-recapture} jest szerokie i~odnosi się do szeregu metod mających na celu oszacowanie wielkości nieznanej populacji. Zwykle wykorzystuje się różnego rodzaju narzędzia statystyczne, na przykład modele log-liniowe, modele klas ukrytych czy uogólnione modele liniowe. W~prezentowanym artykule, celem oszacowania liczby cudzoziemców \DIFdelbegin \DIFdel{na regionalnych rynkach, pracy }\DIFdelend wykorzystane zostały metody \textit{capture-recapture} wykorzystujące analizę log-liniową. Natomiast warto pamiętać, że wybór odpowiedniej techniki w~estymacji liczebności populacji trudnych do zbadania podyktowany jest w~dużej mierze liczbą dostępnych źródeł, którą można podzielić na przypadek wyłącznie jednego albo dwóch lub więcej źródeł. 

Kluczowym aspektem wszystkich metod \textit{capture-recapture} są założenia, których niespełnienie skutkuje obciążonymi szacunkami wielkości populacji. W~przypadku jednego źródła zakładamy, że: (1) jednostki możemy zidentyfikować, (2) jednostki obserwujemy wielokrotnie (na przykład dana osoba popełniła więcej niż jedno przestępstwo), (3) populacja jest stała w~czasie, (4) zakładamy określony rozkład prawdopodobieństwa wielokrotnego wystąpienia w~zbiorze danych (na przykład ucięty rozkład Poissona) oraz (5) niezależność kolejnych obserwacji \citep[por. ][]{van2003point,van2003estimating}. Założenie o~niezależności zdarzeń jest bardzo restrykcyjne i~w~praktyce rzadko możliwe do spełnienia \citep[por. ][]{zhang2008developing}. Dlatego \citet{godwin2017estimation} zaproponowali wykorzystanie dodatniego rozkładu Poissona z~podwyższoną liczbą jedynek do opisu liczby wystąpień w~jednym źródle łagodząc w~ten sposób założenie o~niezależności zdarzeń będącego wynikiem: (1) nauczenia się przez badane jednostki jak być nierozpoznanym/uniknąć złapania lub (2) nieprzyjemności związanych z~pierwszym zdarzeniem i~niechęci do powtórzenia sytuacji. 

W~kontekście dwóch lub więcej źródeł \citet{wolter1986some} zdefiniował następujące założenia: (1) definicje populacji we wszystkich źródłach są takie same (tj. każda jednostka z~populacji ma dodatnie prawdopodobieństwo pojawienia się w~wybranych źródłach), (2) populacja jest zamknięta (tj. stała w~danym czasie), (3) źródła danych są niezależne, (4) brak błędów pokrycia i~duplikatów, (5) brak błędów łączenia między źródłami (tj. łączenie następuje po identyfikatorze) oraz (6) prawdopodobieństwa włączenia do co najmniej jednego z~rejestrów powinny być jednorodne. Spełnienie tych założeń jest kluczowe w~kontekście możliwości stosowania omawianych metod zarówno w~przypadku dwóch, jak i~wielu źródeł. Wrażliwość estymatorów wielkości populacji na złamanie powyższych założeń jest aktualnie poddawane dyskusji w~literaturze poświęconej statystyce publicznej \citep[por. ][]{zhang2015modelling, gerritse2015sensitivity, gerritse2016application, di2015coverage, di2018population, griffin2014potential, zhang-dunne-2017}. W~artykule z~racji ograniczonego miejsca nie podjęto próby oceny wrażliwości na \DIFdelbegin \DIFdel{złamanie }\DIFdelend \DIFaddbegin \DIFadd{niespełnienie powyższych }\DIFaddend założeń\DIFdelbegin \DIFdel{, planowane }\DIFdelend \DIFaddbegin \DIFadd{. Planowane }\DIFaddend jest to w~przyszłych pracach \DIFaddbegin \DIFadd{autorów}\DIFaddend .

W~kontekście statystyki publicznej, metody \textit{capture-recapture} wykorzystuje się do oceny jakości spisów w~ramach tzw. badań pospisowych czy spisów kontrolnych (ang. \textit{post-enumeration surveys} (PES) albo \textit{Census Coverage Survey} (CSS)). W~skrócie, polega to na przeprowadzeniu niezależnego badania reprezentacyjnego w~celu określenia pokrycia spisu. Przykładowo, w~przypadku NSP 2002 oraz 2011 wykorzystano spisy kontrolne, jednakże ich wyniki nie zostały opublikowane \citep[por. ][]{golata2012spis}. 

Metodę \textit{capture-recapture} zaadaptowano także do określenia wielkości populacji wyłącznie na podstawie rejestrów administracyjnych. W~takim \DIFdelbegin \DIFdel{przypadku}\DIFdelend \DIFaddbegin \DIFadd{wypadku}\DIFaddend , metodę \textit{capture-recapture} można znaleźć pod pojęciem dualnej metody estymacji (ang. \textit{dual-system estimation}; DSE) jeżeli wykorzystuje się dwa źródła danych czy potrójnej metody estymacji (ang. \textit{triple-system estimation}; TSE) w~przypadku trzech źródeł. Na przykład, \citet{zhang-dunne-2017} rozważali wykorzystanie metody \textit{capture-recapture} do estymacji populacji Irlandii na podstawie rejestru aktywności osób (ang. \textit{Person Activity Register}) będącego wynikiem łączenia 10 rejestrów administracyjnych według podejścia opartego na znakach życia (ang. \textit{signs-of-life}) oraz ewidencji praw jazdy. \citet{bakker20017} podjęli próbę estymacji liczby niezarejestrowanych rezydentów w~Holandii wykorzystując trzy źródła danych: rejestr ludności, rejestr zatrudnionych oraz rejestr podejrzanych o~przestępstwa prowadzony przez policję. Wykorzystano, celem oszacowania wielkości populacji niezarejestrowanych rezydentów, odpowiednio zbudowany model log-liniowy uwzględniając zmienne pomocnicze w~postaci czasu pobytu, płci oraz wieku, wcześniej dokonując deterministycznego i~probabilistycznego łączenia rekordów z~trzech wspomnianych źródeł danych. Została również przeprowadzona analiza wrażliwości uzyskanych wyników na przypadek występowania błędów połączenia jak i~poprawności procesu parowania jednostek. 

W~literaturze, można znaleźć również wiele innych przykładów wykorzystania omawianej metody do szacunku liczebności specyficznych \DIFdelbegin \DIFdel{subpobulacji}\DIFdelend \DIFaddbegin \DIFadd{subpopulacji}\DIFaddend , na przykład liczby bezdomnych \citep{hudson1998estimating, coumans2017estimating, schepers2017}, narkomanów \citep{van2013methods,Bouch2007, Bouch2008,BouchTrem,Rossi2008}, nietrzeźwych kierowców \citep{van2003estimating}, ofiar konfliktów \citep{chen2018unique} czy liczby cudzoziemców, na której skupimy się w~kolejnej części artykułu. Ciekawy przegląd zastosowań metod \textit{capture-recapture} w~estymacji liczebności populacji trudnych do zbadania można również znaleźć w~pracy \citet{godwin2017estimation}.

\vspace{0.5cm}

\begin{center}
\DIFdelbegin \subsection*{\DIFdel{Metody capture-recapture w~estymacji liczby cudzoziemców}}    
%DIFAUXCMD
\DIFdelend \DIFaddbegin \textbf{\DIFadd{Metody capture-recapture w~estymacji liczby cudzoziemców}}    
\DIFaddend \end{center}

\citet{van2003point} rozważał wykorzystanie jednego źródła danych do estymacji liczby cudzoziemców nielegalnie przebywających w~1995 roku w~Amsterdamie, Rotterdamie, Hadze oraz Utrechcie, którzy nie zostali skutecznie wydaleni z~Holandii. Cudzoziemcy ci byli wielokrotnie obserwowani w~zbiorach danych policji. \citet{van2003point} do estymacji wielkości tak zdefiniowanej populacji zastosowali rozkład Poissona ucięty w~zerze oraz odpowiadający mu uogólniony model liniowy (ang. \textit{zero-truncated Poisson regression model}) wykorzystując następujące zmienne pomocnicze: wiek (do 40, powyżej 40 lat), płeć, narodowość (Turcja, Północna Afryka, reszta Afryki, Surinam, Azja, Ameryka i~Australia) oraz powód wydalenia (nielegalne przebywanie, pozostałe).

\citet{godwin2017estimation} ponownie rozważyli zbiór danych wykorzystany przez \citet{van2003point} ale zakładając, że zdarzenia są zależne, tj. cudzoziemcy raz złapani przez policję mogą nauczyć się w~jaki sposób unikać kolejnego spotkania lub postanowili zalegalizować swój pobyt. W~tym celu autorzy zaproponowali wykorzystanie dodatniego rozkładu Poissona z~podwyższoną liczbą jedynek (pierwszych złapań) oraz uogólnionego modelu liniowego zakładającego ten rozkład dla badanej cechy. Wykorzystanie tego podejścia znacząco obniżyło szacunki wielkości populacji (3 455) w~porównaniu z~podejściem \citet{van2003point} (7 080). Natomiast wykorzystanie zmiennych pomocniczych zwiększyło estymowaną liczbę cudzoziemców nielegalnie przebywających na terenie wyżej wymienionych miast w~1995 roku do odpowiednio 6 272 oraz 12 690. Wydaje się, że w~przypadku wykorzystania jednego źródła danych podejście zaproponowane przez \citet{godwin2017estimation} jest właściwe. 

W~kontekście dwóch i~większej liczby źródeł \citet{van2012people} przedstawili z~kolei interesującą technikę estymacji osób urodzonych na Środkowym Wschodzie (Afganistan, Irak oraz Iran) ale przebywających w~Holandii. W~tym celu wykorzystali modele log-liniowe uwzględniające tzw. pasywne i~aktywne zmienne pomocnicze. W~procesie szacowania tak zdefiniowanej populacji użyte zostały dwa rejestry: rejestr osób, którym wydano zezwolenie na pobyt w~Holandii oraz rejestr policyjny zawierający informacje o~osobach, które podejrzane są o~popełnienie przestępstw. 

Z~kolei \citet{gerritse2015sensitivity} rozważali problem estymacji liczby Polaków oraz osób urodzonych na Środkowym Wschodzie, a~przebywających w~Holandii odpowiednio w~2011 i~2009 roku. W~tym celu autorzy wykorzystali, podobnie jak \citet{van2012people}, dwa rejestry administracyjne -- rejestr osób zameldowanych w~Holandii oraz rejestr policyjny -- skupiając się jednak na wrażliwości estymatora wielkości populacji opartego na modelach log-liniowych na złamanie założenia o~niezależności tych dwóch źródeł. W~przypadku osób urodzonych na Bliskim Wschodzie wpływ złamania założeń \textit{capture-recapture} jest niewielki, podczas gdy dla obywateli Polski różnice w~wielkości populacji są bardzo znaczące. W~swojej rozprawie doktorskiej \citet{gerritse2016application} analizowała problemy niespełnienia założeń metody \textit{capture-recapture} (zależności źródeł oraz błędów w~łączeniu rekordów) oraz wpływu imputacji danych na estymację liczby rezydentów według czasu przebywania. W~tym celu, oprócz rejestru ludności i~policji, wykorzystała rejestr osób zatrudnionych. 

Ciekawą alternatywę dla wykorzystania danych jednostkowych z~wielu źródeł zaproponował \citet{zhang2008developing} w~kontekście estymacji subpopulacji cudzoziemców. Na potrzeby estymacji wielkości populacji odnoszącej się do nielegalnie przebywających w~Norwegii cudzoziemców\footnote{Autor w~swojej pracy używał zamiennie  pojęć \textit{unauthorized foreigners} oraz \textit{irregular foreigners} w~kontekście rezydentów, którzy przebywali na terenie Norwegii bez odpowiednich dokumentów umożliwiających ich pobyt.}, \citeauthor{zhang2008developing} wykorzystał trzy źródła danych. Pierwsze źródło stanowił Centralny Rejestr Osób (Central Personel Register), z~którego wykorzystano informacje na temat liczby zameldowanych osób urodzonych poza Norwegią według kraju urodzenia i~w~wieku 18+. Drugim z~wykorzystanych źródeł były dane na temat liczby obcokrajowców według kraju obywatelstwa, którzy zostali oskarżeni o~popełnienie przestępstwa. Tego typu informacje dostarcza Krajowy Urząd Statystyczny w~Norwegii. Ostatnim źródłem danych był rejestr DUF (nor. \textit{Datasystemet for utlendings og flyktningsaker}), w~którym znajdują się wszystkie osoby ubiegające się o~zamieszkanie w~Norwegii. Jest to baza obejmująca imigrantów i~uchodźców, którym przyznawany jest 12-cyfrowy numer w~momencie ubiegania się przez nich o~możliwość zamieszkania w~Norwegii. Z~tego źródła wykorzystano informację o~liczbie wniosków o~wydalenie z~Norwegii uwzględniając podział czy dane osoby wnioskowały o~azyl. Na potrzeby estymacji wielkości populacji nielegalnie przebywających w~Norwegii cudzoziemców wykorzystano hierarchiczny model gamma Poissona, który należy do rodziny modeli mieszanych z~efektami losowymi. W~charakterze efektu losowego wykorzystano kraj pochodzenia cudzoziemców.

Na podstawie powyższych rozważań należy zauważyć pewną powtarzalność w~kontekście doboru źródeł danych. Podstawą wszystkich estymacji było wykorzystanie rejestru osób (populacji \textit{de \DIFdelbegin \DIFdel{jure}\DIFdelend \DIFaddbegin \DIFadd{iure}\DIFaddend }) oraz danych pochodzących z~policji. Główną przesłanką takiego wyboru jest spełnienie założenia o~niezależności źródeł danych. Dlatego, aby poprawnie oszacować wielkość populacji, kluczowe jest dobranie odpowiednich zbiorów administracyjnych celem spełnienia tego kluczowego założenia, od którego zależy zasadność stosowania metod \textit{capture-recapture}. 

Powyżej przytoczone przykłady wskazywały na praktyczne wykorzystanie metod \textit{ca\-ptu\-re-recapture} bazujących na modelach log-liniowych czy Poissona w~estymacji liczby cudzoziemców w~innych krajach. W~przypadku Polski brak jest kompleksowych opracowań skupiających się na estymacji faktycznej liczby cudzoziemców. Po części może wynikać to z~faktu, że dopiero w~ostatnich latach tematyka cudzoziemców w~Polsce (zwłaszcza osób pochodzących z~Ukrainy) nabrała dużego znaczenia, zwłaszcza w~kontekście rynku pracy. Warto jednak podkreślić, że pewne próby estymacji dokonują pracownicy Narodowego Banku Polskiego na podstawie danych zagregowanych na potrzeby modelu NECMOD (ekonometrycznego modelu polskiej gospodarki) oraz szacunków przekazów pieniężnych. Także w~mediach pojawiają się różne szacunki, które w~żaden sposób nie są weryfikowalne. Rejestr PESEL zawiera bowiem wyłącznie osoby, które są zameldowane na pobyt czasowy lub stały, Zakład Ubezpieczeń Społecznych dysponuje liczbą cudzoziemców zgłoszonych do ubezpieczenia, Ministerstwo Rodziny, Pracy i~Polityki Społecznej z~kolei \DIFdelbegin \DIFdel{informacjami }\DIFdelend \DIFaddbegin \DIFadd{dysponuje danymi }\DIFaddend o~chęci zatrudnienia cudzoziemców, Urząd ds. Cudzoziemców posiada \DIFdelbegin \DIFdel{informacje }\DIFdelend \DIFaddbegin \DIFadd{dane }\DIFaddend dotyczące ubiegania się o~wizy czy karty pobytu, Straż Graniczna dostarcza statystyk dotyczących m.in. ruchu granicznego czy liczby cudzoziemców nielegalnie przebywających na terenie Polski, a~Policja dysponuje Krajowym Systemem Informacji, który zawiera dane o~popełnionych przestępstwach. Wydaje się jednak, że polska statystyka publiczna wspierana zasobami informacyjnymi pochodzącymi od innych organów, dysponuje wszelkimi zbiorami umożliwiającymi podjęcie rzetelnej próby szacunku liczby cudzoziemców. Do tej pory, zgodnie z~aktualną wiedzą autorów, w~Polsce nie było jednak podejmowanych prób estymacji liczby cudzoziemców z~wykorzystaniem wyżej rozważanych metod. Niniejszy artykuł oraz wspomniany na wstępie projekt badawczy wychodzą naprzeciw oczekiwaniom wielu odbiorców odnośnie informacji o~liczbie cudzoziemców w~Polsce. 


%%%%%%%%%%%%%%%%%%%%%%%%%%%%%%%%%%%%%%%%%%%%%%%%%%%%%%%%%%%%%%%%%%%%%%%%%%%%%%%%%%%%%%%%%%%%%

\begin{center}
\DIFdelbegin \section*{\DIFdel{MODELE LOG-LINIOWE W~SZACOWANIU WIELKOŚCI POPULACJI TRUDNYCH DO ZBADANIA}}
%DIFAUXCMD
\DIFdelend \DIFaddbegin \DIFadd{MODELE LOG-LINIOWE W~SZACOWANIU WIELKOŚCI POPULACJI TRUDNYCH DO ZBADANIA
}\DIFaddend \end{center}

Na potrzeby estymacji liczby cudzoziemców w~Polsce z~uwzględnieniem dodatkowych przekrojów zdecydowano się wykorzystać metodę \textit{capture-recapture} bazującą na modelach log-liniowych. Wynikało to przede wszystkim z~dostępności odpowiednich źródeł danych, które można wykorzystać w~tego typu szacunkach, odpowiednich pakietów programu R, w~których zaimplementowane są funkcje na potrzeby estymacji parametrów modeli log-liniowych oraz kodów na procedurę bootstrap umożliwiającą znalezienie właściwych przedziałów ufności, a~także z~faktu, że w~literaturze przedmiotu właśnie te modele są z~powodzeniem wykorzystywane w~estymacji liczebności populacji trudnych do zbadania. Przykład stanowić mogą wspomniane już prace \citet{coumans2017estimating} oraz \citet{van2012people}. W~ pierwszej z~prac wykorzystano modele log-liniowe do oszacowania liczby bezdomnych osób w~Holandii. W~drugim z~przytoczonych artykułów zastosowanie modeli log-liniowych oraz koncepcji pasywnych i~aktywnych zmiennych pomocniczych umożliwiło oszacowanie liczby osób urodzonych na Bliskim Wschodzie a~przebywających w~Holandii.

Modele log-liniowe stanowią obecnie bardzo ważną metodę analizy danych zawartych w~tablicach kontyngencji. Rozwój metodologii dedykowanej tej technice analizy danych jakościowych zapoczątkowany został w~latach 60-tych XX wieku. \citet{goodman1964,goodman1968,goodman1969} był jednym z~pierwszych badaczy, którzy spopularyzowali modele log-liniowe w~naukach społecznych. Modele te są szczególnie przydatne w~sytuacjach, gdy brak jest precyzyjnego rozróżnienia między zmienną objaśnianą a~zmiennymi objaśniającymi, a~zachodzi potrzeba wykrycia zależności w~pewnym zbiorze danych.

Punktem wyjścia do zastosowania modeli log-liniowych w~estymacji liczebności populacji trudnych do zbadania jest odpowiednio skonstruowana tablica kontyngencji\footnote{Na potrzeby szacunku liczby cudzoziemców rozpatrywane były złożone tablice wielowymiarowe. Celem przedstawienia idei modeli log-liniowych w~tym zagadnieniu, w~artykule ograniczymy się do tablic typu $2\times 2$ oraz $2\times 2\times 2$.}, w~której wykorzystuje się informacje z~dwóch lub większej liczby źródeł danych. W~Tabeli \ref{tab1} przedstawiono przypadek, gdy dysponujemy dwoma niezależnymi źródłami danych (powiedzmy A i~B). Tabela taka powstaje poprzez połączenie informacji o~populacji trudnej do zbadania z~dwóch różnych źródeł.

\begin{table}[ht]
\centering
\caption{Przypadek dwóch źródeł - tablica kontyngencji $2\times 2$}\label{tab1}
\begin{tabular}{l|lll|l}
\hline
\multicolumn{1}{c|}{} & \multicolumn{1}{c}{} & \multicolumn{2}{c|}{Źródło B} & \multicolumn{1}{c}{} \\ 
\hline
\multicolumn{1}{c|}{} & \multicolumn{1}{c}{} & \multicolumn{1}{c}{Tak (1)} & \multicolumn{1}{c|}{Nie (0)} & \multicolumn{1}{c}{$\sum$} \\ 
\multicolumn{1}{c|}{Źródło A} & \multicolumn{1}{c}{Tak (1)} & \multicolumn{1}{c}{$n_{11}$} & \multicolumn{1}{c|}{$n_{10}$} & \multicolumn{1}{c}{$n_{1+}$} \\ 
\multicolumn{1}{c|}{} & \multicolumn{1}{c}{Nie (0)} & \multicolumn{1}{c}{$n_{01}$} & \multicolumn{1}{c|}{$n_{00}$} & \multicolumn{1}{c}{$n_{0+}$} \\ 
\hline
\multicolumn{1}{c|}{$\sum$} & \multicolumn{1}{c}{} & \multicolumn{1}{c}{$n_{+1}$} & \multicolumn{1}{c|}{$n_{+0}$} & \multicolumn{1}{c}{$n$} \\ 
\hline
\end{tabular}
\legend{Źródło: opracowanie własne}
\end{table}


W~przypadku dwóch źródeł danych A i~B może mieć miejsce sytuacja, w~której po połączeniu jednostek\footnote{W~tym celu można wykorzystać łączenie deterministyczne z~wykorzystaniem odpowiedniego identyfikatora lub probabilistyczne łączenie rekordów.} występują jednostki tylko w~źródle A, a~nie występują w~źródle B, występują w~źródle B i~nie występują w~źródle A oraz występują jednocześnie w~źródle A i~B. W~powyższej tabeli Tak (1) oznacza, że dana jednostka występuje w~odpowiednim źródle, a~Nie (0), że jednostka w~tym źródle nie występuje. Przykładowo, $n_{01}$ oznacza liczbę jednostek, które nie występują w~źródle A, a~występują w~źródle B. Kluczową kwestią jest zatem oszacowanie liczebności $n_{00}$, tj. liczby jednostek, które nie występują zarówno w~źródle A jak i~B. Ostatecznie oszacowaną liczebność populacji uzyskuje się bowiem poprzez dodanie wszystkich wartości z~Tabeli \ref{tab1} po wcześniejszej estymacji liczebności $n_{00}$. 

Oszacowanie liczebności $n_{00}$ może być uzyskane poprzez dopasowanie modelu log-liniowego do niekompletnej tablicy kontyngencji. Przykładowo, dla Tabeli \ref{tab1} wymiarów $2\times 2$ odnoszących się do źródeł danych A i~B pełen model log-liniowy [AB]\footnote{Jest to tzw. notacja nawiasowa, która w~przypadku modeli log-liniowych jest często stosowana.} może być przedstawiony w~postaci (ang. \textit{saturated model}):

\begin{equation}
\ln \left(m_{ij}\right)=\mu +\lambda^{A}_{i}+\lambda^{B}_{j}+\lambda^{AB}_{ij}, \quad i,j=\left\{\mathrm{'Tak'},\mathrm{'Nie'}\right\},
\end{equation}

\noindent gdzie $m_{ij}$ oznacza oczekiwaną liczebność w~komórce $i,j$. Ponieważ jednak komórka $m_{00}=m_{(\textrm{Nie},\textrm{Nie})}$ nie jest obserwowana model [AB] ma jeden parametr za dużo i~nie może być zatem estymowany. W~takiej sytuacji można rozważyć model niezależności [A][B] postaci:

\begin{equation}
\ln \left(m_{ij}\right)=\mu +\lambda^{A}_{i}+\lambda^{B}_{j},
\end{equation}

\noindent który ma tylko trzy parametry do oszacowania w~związku z~brakiem efektu interakcji $\lambda^{AB}_{ij}$. Ponieważ mamy trzy obserwowane komórki w~Tabeli \ref{tab1} oraz trzy parametry do oszacowania mamy w~zasadzie do czynienia z~modelem nasyconym. Po dopasowaniu tego modelu do danych możemy użyć oszacowanych parametrów do wyznaczenia liczebności brakującej komórki (‘Nie’, ‘Nie’), a~następnie wyznaczyć liczebność populacji poddanej analizie. Oszacowanie liczebności komórki $n_{00}$ znajdujemy przy tym ze wzoru:

\begin{equation}
\hat{n}_{00}=\exp\left(\mu\right).
\end{equation}

Podobne rozumowanie można przeprowadzić w~odniesieniu do tablic trójdzielczych typu $2\times 2\times 2$, tj. w~sytuacji, gdy dysponujemy trzema źródłami danych A,B i~C. 

\begin{table}[ht]
\centering
\caption{Przypadek trzech źródeł - tablica kontyngencji $2\times 2\times 2$}\label{tab2}
\begin{tabular}{l|llllll}
\hline
\multicolumn{1}{c|}{} & \multicolumn{1}{c}{} & \multicolumn{4}{c}{Źródło C} & \multicolumn{1}{c}{} \\ 
\hline
\multicolumn{1}{c|}{} & \multicolumn{1}{c}{} & \multicolumn{2}{c}{Źródło B} & \multicolumn{2}{c}{Źródło B} & \multicolumn{1}{c}{} \\ 
\hline
\multicolumn{1}{c|}{} & \multicolumn{1}{c}{} & \multicolumn{1}{c}{Tak (1)} & \multicolumn{1}{c}{Nie (0)} & \multicolumn{1}{c}{Tak (1)} & \multicolumn{1}{c}{Nie (0)} & \multicolumn{1}{c}{$\sum$} \\ 
\multicolumn{1}{c|}{Źródło A} & \multicolumn{1}{c}{Tak (1)} & \multicolumn{1}{c}{$n_{111}$} & \multicolumn{1}{c}{$n_{101}$} & \multicolumn{1}{c}{$n_{110}$} & \multicolumn{1}{c}{$n_{100}$} & \multicolumn{1}{c}{$n_{1++}$} \\ 
\multicolumn{1}{c|}{} & \multicolumn{1}{c}{Nie (0)} & \multicolumn{1}{c}{$n_{011}$} & \multicolumn{1}{c}{$n_{001}$} & \multicolumn{1}{c}{$n_{010}$} & \multicolumn{1}{c}{$n_{000}$} & \multicolumn{1}{c}{$n_{0++}$} \\ 
\hline
\multicolumn{1}{c|}{$\sum$} & \multicolumn{1}{c}{} & \multicolumn{1}{c}{$n_{+11}$} & \multicolumn{1}{c}{$n_{+01}$} & \multicolumn{1}{c}{$n_{+10}$} & \multicolumn{1}{c}{$n_{+00}$} & \multicolumn{1}{c}{n} \\ 
\hline
\end{tabular}
\legend{Źródło: opracowanie własne}
\end{table}

Tabela \ref{tab2} może przedstawiać sytuację trzech źródeł, na przykład trzech rejestrów administracyjnych, dwóch rejestrów administracyjnych i~badania reprezentacyjnego czy spisu. Podobnie jak w~przypadku tabeli $2\times 2 \times 2$ istotne jest określenie przynależności do poszczególnego źródła (oznaczone jako Tak/Nie). Również i~w~tym przypadku chcemy oszacować to czego nie możemy odczytać z~tabeli, tj.  $n_{000}$. Na potrzeby estymacji liczebności $n_{000}$ można również wykorzystać koncepcję modeli log-liniowych. W~tym celu budujemy model log-liniowy postaci (bez efektu głównego $\lambda_{ijk}^{ABC}$) :

\begin{equation}
\ln \left(m_{ij}\right)=\mu +\lambda^{A}_{i}+\lambda^{B}_{j}+\lambda^{C}_{k}+\lambda^{AB}_{ij}+\lambda^{AC}_{ik}+\lambda^{BC}_{jk},
\end{equation}

\noindent który musimy ograniczyć przez:
$\lambda_{0}^{A}=\lambda_{0}^{B}=\lambda_{0}^{C}=\lambda_{00}^{AB}=\lambda_{10}^{AB}=\lambda_{01}^{AB}=\lambda_{00}^{AC}=\lambda_{10}^{AC}=\lambda_{01}^{AC}=\lambda_{00}^{BC}=\lambda_{10}^{BC}=\lambda_{01}^{BC}=0$, aby móc oszacować parametry. Dodatkowym założeniem jest to, że nie występuje interakcja między A, B i~C, tj. $\lambda_{ijk}^{ABC}=0$. Model ten w~notacji nawiasowej oznacza się przez [AB][BC][AC]. Oszacowanie brakującej liczby jednostek populacji otrzymujemy ze wzoru:

\begin{equation}
\hat{n}_{000}=\exp\left(\mu\right),
\end{equation}

\noindent po uprzednim wyestymowaniu wszystkich parametrów.

W~przypadku estymacji wielkości populacji możliwe jest wykorzystanie zmiennych pomocniczych, którymi mogą być przykładowo płeć czy grupy wieku. Celem jest z~jednej strony obejście jednego z~założeń metody \textit{capture-recapture} (o~stałej stopie pokrycia przez źródło w~populacji) i~uwzględnienie faktu heterogeniczności przynależności poszczególnych jednostek do źródeł. Wykorzystanie zmiennych pomocniczych w~kontekście modeli log-liniowych rozważa m.in. \citet{gerritse2016application}, \citet{coumans2017estimating}, \citet{van2012people} czy \citet{zwane2005population}. Wyróżniamy przy tym dwa podejścia, które determinowane są dostępnością zmiennych we wszystkich, niektórych lub tylko w~jednym źródle. Pierwsze określa się w~literaturze jako podejście z~pełni obserwowalnymi zmiennymi (ang. \textit{fully observed covariates}), a~drugie z~częściowo obserwowalnymi zmiennymi (ang. \textit{partially observed covariates}). W~obydwu przypadkach można wykorzystać modele log-liniowe do oszacowania poszczególnych elementów populacji. Tego typu podejście zostało również zastosowane na potrzeby tego artykułu. Przykładowo, w~przypadku dwuwymiarowej tabeli kontyngencji $2\times 2$ oprócz przynależności do dwóch źródeł A i~B można rozpatrywać dodatkową cechę X (na przykład płeć) przez co należy rozszerzyć tabelę do trójdzielczej Tabeli \ref{tab3} oraz dopasować model log-liniowy [AX][BX] postaci:

\begin{equation}
\ln \left(m_{ijx}\right)=\mu +\lambda^{A}_{i}+\lambda^{B}_{j}+\lambda^{X}_{x}+\lambda^{AX}_{ix}+\lambda^{BX}_{jx},\label{model}
\end{equation}

\noindent gdzie $\lambda_{ix}^{AX}$ oraz $\lambda_{jx}^{BX}$ oznaczają efekty interakcji pomiędzy zmienną pomocniczą X i~źródłami danych A oraz B odpowiednio. 

\begin{table}[ht]
\centering
\caption{Przypadek dwóch źródeł A i~B oraz jednej zmiennej pomocniczej X}\label{tab3}
\begin{tabular}{l|llllll}
\hline
\multicolumn{1}{c|}{} & \multicolumn{1}{c}{} & \multicolumn{4}{c}{Zmienna X} & \multicolumn{1}{c}{} \\ 
\hline
\multicolumn{1}{c|}{} & \multicolumn{1}{c}{} & \multicolumn{2}{c}{$X_{1}$} & \multicolumn{2}{c}{$X_{2}$} & \multicolumn{1}{c}{} \\ 
\hline
\multicolumn{1}{c|}{} & \multicolumn{1}{c}{} & \multicolumn{2}{c}{Źródło B} & \multicolumn{2}{c}{Źródło B} & \multicolumn{1}{c}{} \\ 
\hline
\multicolumn{1}{c|}{} & \multicolumn{1}{c}{} & \multicolumn{1}{c}{Tak (1)} & \multicolumn{1}{c}{Nie (0)} & \multicolumn{1}{c}{Tak (1)} & \multicolumn{1}{c}{Nie (0)} & \multicolumn{1}{c}{$\sum$} \\ 
\multicolumn{1}{c|}{Źródło A} & \multicolumn{1}{c}{Tak (1)} & \multicolumn{1}{c}{$n_{111}$} & \multicolumn{1}{c}{$n_{101}$} & \multicolumn{1}{c}{$n_{110}$} & \multicolumn{1}{c}{$n_{100}$} & \multicolumn{1}{c}{$n_{1++}$} \\ 
\multicolumn{1}{c|}{} & \multicolumn{1}{c}{Nie (0)} & \multicolumn{1}{c}{$n_{011}$} & \multicolumn{1}{c}{$n_{001}$} & \multicolumn{1}{c}{$n_{010}$} & \multicolumn{1}{c}{$n_{000}$} & \multicolumn{1}{c}{$n_{0++}$} \\ 
\hline
\multicolumn{1}{c|}{$\sum$} & \multicolumn{1}{c}{} &  \multicolumn{1}{c}{$n_{+11}$} & \multicolumn{1}{c}{$n_{+01}$} & \multicolumn{1}{c}{$n_{+10}$} & \multicolumn{1}{c}{$n_{+00}$} & \multicolumn{1}{c}{n} \\ 
\hline
\end{tabular}
\legend{Źródło: opracowanie własne}
\end{table}

W~przypadku dwóch źródeł A i~B oraz jednej zmiennej pomocniczej X, przyjmującej przykładowo dwa warianty $X_{1}$ oraz $X_{2}$ (na przykład mężczyzna i~kobieta), mamy do czynienia z~trójdzielczą tablicą kontyngencji $2\times 2 \times 2$, w~której brakujące liczebności podlegające estymacji to $n_{001}$ oraz $n_{000}$. Mamy zatem sześć komórek, dla których znane są obserwowane liczebności w~Tabeli \ref{tab3}, w~związku z~czym model (\ref{model}) zawiera sześć parametrów, które należy oszacować (nasycony model log-liniowy).  Po dopasowaniu modelu do danych brakujące liczebności komórek znajdujemy ze wzorów:  $\hat{n}_{000}=\exp\left(\mu\right)$ oraz  $\hat{n}_{001}=\exp\left(\mu+\lambda^{X}_{X_{1}}\right)$. Powyższe rozumowanie w~naturalny sposób można rozszerzyć na większą liczbę zmiennych pomocniczych oraz liczbę analizowanych źródeł. Zwiększa się przez to w~oczywisty sposób złożoność analizowanych modeli log-liniowych, jednak wykorzystanie odpowiednich pakietów \DIFaddbegin \DIFadd{(np. }\texttt{\DIFadd{stats}} \DIFadd{i }\texttt{\DIFadd{parallel}}\DIFadd{) }\DIFaddend języka R \DIFaddbegin \DIFadd{\mbox{%DIFAUXCMD
\citep{r-cran} }\hspace{0pt}%DIFAUXCMD
}\DIFaddend znacznie skraca proces estymacji wszystkich możliwych do zbudowania modeli.         

%%%%%%%%%%%%%%%%%%%%%%%%%%%%%%%%%%%%%%%%%%%%%%%%%%%%%%%%%%%%%%%%%%%%%%%%%%%%%%%%%%%%%%%%%%%%%

\begin{center}
\DIFdelbegin \section*{\DIFdel{METODY OCENY JAKOŚCI MODELI LOG-LINIOWYCH}}
%DIFAUXCMD
\DIFdelend \DIFaddbegin \DIFadd{METODY OCENY JAKOŚCI MODELI LOG-LINIOWYCH
}\DIFaddend \end{center}

W~analizie log-liniowej głównym celem jest wybór modelu o~możliwie najprostszej postaci, który jednocześnie byłby najlepiej dopasowany do danych. W~literaturze przedmiotu \citep{goodman1964,goodman1968,goodman1969,brzezinska2015analiza} proponuje się na potrzeby oceny modeli różnego rodzaju kryteria. Zostały one również wykorzystane na potrzeby artykułu w~procesie wyboru i~oceny finalnego modelu. Do najważniejszych kryteriów zaliczamy iloraz wiarogodności, dewiancję, AIC oraz BIC. % oraz współczynniki determinacji $R^{2}$.

Iloraz wiarogodności jest miarą pozwalającą ocenić dopasowanie modelu do danych. Przykładowo dla tablic $2\times 2$ wyraża się on wzorem:

\begin{equation}
G^{2}=2\sum_{i=1}^{2}\sum_{j=1}^{2}n_{ij}\ln\left(\frac{n_{ij}}{\hat{m}_{ij}}\right),
\end{equation}

\noindent gdzie $\hat{m}_{ij}=\frac{n_{i+}n_{+j}}{n}$ stanowią oszacowania liczebności teoretycznych wyznaczonych dla danego modelu log-liniowego. W~sytuacji, gdy wartość ilorazu wiarogodności $G^2$ jest duża to wówczas model taki powinien być odrzucony jako model, który w~nieprawidłowy sposób odwzorowuje zależności między badanymi zmiennymi. Współczynnik $G^2$ może być także wykorzystywany do porównania oceny różnych modeli. W~sytuacji, gdy porównujemy dwa modele współczynnik $G^2$ może zostać przedstawiony w~postaci (dla tablic $2\times2$):

\begin{equation}
G^{2}=2\sum_{i=1}^{2}\sum_{j=1}^{2}\hat{m}_{ij}^{0}\ln\left(\frac{\hat{m}_{ij}^{0}}{\hat{m}_{ij}^{1}}\right),
\end{equation}

\noindent gdzie: 0 odnosi się do liczebności teoretycznych modelu ogólniejszego, tj. zawierającego wszystkie możliwe parametry, natomiast 1 dotyczy liczebności teoretycznych modelu zagnieżdżonego o~uproszczonej postaci i~zawierającego się w~modelu 0. Współczynnik ten może być również przedstawiony w~postaci:

\begin{equation}
G^2\left(M_0 |M_1\right)=G^2\left(M_0\right)-G^2\left(M_1\right). \end{equation}

Powyższa statystyka ma rozkład chi-kwadrat o~liczbie stopni swobody $df=df\left(M_0\right)-df\left(M_1\right)$, gdzie $M_0$ jest modelem zagnieżdżonym, a~$M_1$ modelem ogólnym z~większą liczbą parametrów i~nazywana jest dewiancją. Dewiancja pozwala ocenić czy parametr występujący w~modelu $M_1$, a~niewystępujący w~modelu $M_0$ jest statystycznie istotny.

Statystyką służącą do porównywania ze sobą większej liczby modeli jest tzw. kryterium informacyjne Akaike oraz Schwarza (bayesowskie). Kryterium informacyjne Akaike wyraża się wzorem:

\begin{equation}
AIC=G^2-df,   
\end{equation}

\noindent gdzie $G^2$ to iloraz wiarogodności badanego modelu, a~$df$ to liczba odpowiadających mu stopni swobody. Z~kolei bayesowskie kryterium informacyjne wyraża się wzorem:

\begin{equation}
BIC=G^2-df\cdot\ln\left(n\right),    
\end{equation}

\noindent gdzie $n$ to liczebność w~tablicy kontyngencji. Preferowane są przy tym modele, dla których miary $AIC$ i~$BIC$ przyjmują mniejsze wartości. \DIFaddbegin \DIFadd{W pracy wykorzystano kryterium $BIC$ do określenia najlepszego modelu.
}\DIFaddend 

%W~przypadku modeli log-liniowych do ich oceny wykorzystuje się również tzw. współczynnik determinacji $R^2$ (pseudo $R^2$). Wyraża się on wzorem:

%\begin{equation}
%R^2=\frac{G^2\left(M_0\right)-G^2\left(M_1\right)}{G^2\left(M_0\right)}.
%\end{equation}

%Wyższe wartości tego współczynnika są bardziej pożądane. Nie można go jednak wykorzystać do porównywania modeli o~różnej liczbie stopni swobody z~tego powodu, że model o~większej złożoności będzie miał współczynnik pseudo $R^2$ większy w~porównaniu z~modelem prostszym. 

%Alternatywnie do oceny modelu można wykorzystać tzw. skorygowany współczynnik determinacji McFaddena (ang. \textit{adjusted pseudo $R^2$}), który wyraża się wzorem:

%\begin{equation}
%R_{Adj}^{2}=1-\frac{q-r_{0}}{q-r_{1}}\left(1-R^{2}\right),
%\end{equation}

%\noindent gdzie $q$ to liczba komórek w~tablicy kontyngencji (dla dwuwymiarowej tablicy $2\times 2$ mamy $q=4$) a~$r_0$ i~$r_1$ to liczba parametrów odpowiednio w~przypadku modelu $M_0$ i~$M_1$. Im większa jest wartość $R_{Adj}^{2}$, tym model $M_1$ jest lepiej dopasowany do danych. Współczynnik ten nie jest unormowany w~przedziale $\left[0,1\right]$ i~może przyjmować wartości ujemne. Oprócz dwóch wyżej wspomnianych współczynników determinacji można użyć wielu innych miar dopasowania, na przykład współczynnik determinacji $R^{2}$ Efrona\footnote{Informacje na jego temat (oraz wielu innych miar dopasowania $R^{2}$) można znaleźć pod adresem \url{http://www.glmj.org/archives/articles/Smith_v39n2.pdf}.}.

%%%%%%%%%%%%%%%%%%%%%%%%%%%%%%%%%%%%%%%%%%%%%%%%%%%%%%%%%%%%%%%%%%%%%%%%%%%%%%%%%%%%%%%%%%%%%

\begin{center}
\section*{PRECYZJA OSZACOWAŃ LICZEBNOŚCI POPULACJI TRUDNYCH DO ZBADANIA}
\end{center}

Kluczową kwestią w~zagadnieniu estymacji liczebności populacji trudnej do zbadania jest jakość uzyskanych wyników.  Prace nad oceną precyzji oszacowań uzyskanych w~oparciu o~techniki \textit{capture-recapture} prowadzone były przez wielu badaczy oraz instytucji. Przykładowo, Międzynarodowa Grupa Robocza ds. Monitorowania i~Prognozowania Chorób, podjęła prace nad konstrukcją niesymetrycznych przedziałów ufności dla liczebności populacji trudnych do zbadania (International Working Group for Disease Monitoring and Forecasting, 1995). \citet{chao1989estimating} podjął z~kolei próbę konstrukcji symetrycznych przedziałów ufności polegającą na odpowiedniej transformacji oszacowanej liczebności populacji, głównie z~wykorzystaniem transformacji logarytmicznej. Wreszcie ostatnio stosowane techniki w~konstrukcji estymatorów wariancji liczebności populacji trudnych do zbadania bazują na metodzie bootstrap, zarówno nieparametrycznej jak i~parametrycznej \citep{buckland1991quantifying,gemmell2004capture}.

W~artykule na potrzeby oceny jakości oszacowań liczby cudzoziemców w~odpowiednich przekrojach dokonano konstrukcji 95\% przedziałów ufności oraz względnych błędów szacunku. W~tym celu wykorzystano parametryczny bootstrap, który jest szeroko stosowany w~badaniach poświęconych estymacji populacji trudnych do zbadania \citep{zwane2003implementing}. Decyzja o~konstrukcji odpowiednich przedziałów ufności oraz względnych błędów szacunku bazujących na parametrycznej metodzie bootstrap wynikała również z~faktu, że jest to stosunkowo łatwa w~implementacji technika w~kontekście tablic kontyngencji, które nie są w~pełni obserwowalne (nieznajomość liczebności niektórych komórek). Ogólnie, celem utworzenia przedziałów ufności oraz wyznaczenia względnych błędów szacunku, w~pierwszej kolejności dokonuje się oszacowania liczebności populacji trudnej do zbadania z~wykorzystaniem odpowiedniego modelu log-liniowego. Estymację parametrów modelu log-liniowego przeprowadza się przy tym na obserwowalnych komórkach tablicy kontyngencji. Mając oszacowane parametry modelu oraz liczebności brakujących komórek wyznaczane są prawdopodobieństwa teoretyczne przynależności dla wszystkich komórek w~tablicy kontyngencji. Następnie losowana jest próba z~rozkładu wielomianowego przy uwzględnieniu oszacowanych prawdopodobieństw, która w~dalszym etapie jest korygowana, tak aby odpowiadała strukturze obserwowanych danych. Wówczas dokonuje się dopasowania odpowiedniego modelu log-liniowego do kompletnej tablicy kontyngencji i~uzyskuje pierwsze oszacowanie bootstrapowe liczebności populacji trudnej do zbadania. Procedurę tą przeprowadza się wielokrotnie, wyznaczając wariancję, a~następnie przedział ufności dla liczebności populacji. 

Ujmując zagadnienie bardziej formalnie, sposób wyznaczania względnych błędów szacunku oraz przedziałów ufności liczebności populacji trudnej do zbadania w~parametrycznej metodzie boostrap można zapisać w~następujących krokach:

\begin{enumerate}
    \item Dla zadanej tablicy kontyngencji i~komórek, dla których istnieją wartości empiryczne, dokonaj oszacowania parametrów odpowiedniego modelu log-liniowego. 
    \item Wykorzystując parametry wyznaczonego modelu log-liniowego dokonaj oszacowania wielkości populacji we wszystkich założonych przekrojach. 
    \item Wyznacz całkowitą liczebność populacji trudnej do zbadania $\hat{N} = \hat{N}_1 +\ldots+ \hat{N}_P$, gdzie $P$ to liczba komórek w~tablicy kontyngencji, a~$\hat{N}_p$ to oszacowana wielkość populacji w~komórce $p$, przy czym $p=1,\ldots,P$.
    \item Wyznacz wektor długości $P$ złożony z~prawdopodobieństw $\hat{\boldsymbol{\pi}}_P = (\hat{N}_1 / \hat{N},\ldots, \hat{N}_P / \hat{N})^T$.
   \item Wygeneruj z~rozkładu wielomianowego wektor $\boldsymbol{N}^* = ({N}^*_1,\ldots, {N}^*_P)^T$ długości $P$ odpowiadający  populacji o~liczebności $\hat{N}$ z~prawdopodobieństwami $\hat{\boldsymbol{\pi}}_P$. Jest to wektor złożony z~pseudo-liczebności populacji we wszystkich założonych $P$ przekrojach.
    \item Na bazie uzyskanych pseudo-liczebności stwórz tablicę kontyngencji układem odpowiadającą wyjściowej tablicy. Oszacuj parametry modelu log-liniowego dla tych samych komórek co w~punkcie 1.
    \item Dokonaj oszacowania dla przekrojów nieobserwowanych w~tablicy kontyngencji.
    \item Powtórz kroki 5--7 $B$ razy\footnote{Na potrzeby artykułu przyjęto $B=500$.}.
    \item Oszacuj liczebność populacji $\hat{N}^{b}$, dla $b=1,\ldots,B$ oraz liczebności we wszystkich przekrojach rozważanej tablicy kontyngencji.   
    \item Na podstawie otrzymanych oszacowań wyznacz wartość oczekiwaną, wariancję, względny błąd szacunku oraz 95\% przedział ufności liczebności populacji trudnej do zbadania\footnote{W~podobny sposób można wyznaczyć te miary dla liczebności populacji w~odpowiednich przekrojach.}:
    \begin{itemize}
        \item wartość oczekiwana:
    \begin{equation}
    \hat{N}=\frac{\sum_{b=1}^{B}\hat{N}^{B}}{B},    
    \end{equation}
      \item wariancja empiryczna:
    \begin{equation}
    \hat{V}\left(\hat{N}\right)=\frac{1}{B-1}\sum_{b=1}^{B}\left(\hat{N}^{B}-\hat{N}\right)^{2},   
    \end{equation}
      \item względny błąd szacunku (precyzja):
    \begin{equation}
    REE\left(\hat{N}\right)= \frac{\sqrt{\hat{V}\left(\hat{N}\right)}}{\hat{N}},
    \end{equation}
       \item 95\% przedział ufności\footnote{Przedział ten wyznaczany jest metodą percentylową. Na przykład, 95\% percentylowy przedział ufności ma dolną i~górną granicę wyznaczoną przez 2,5 i~97,5 percentyl wartości bootstrapowych $\hat{N}^{B}$.}: 
    \begin{equation}
    \left[\hat{N}_{2,5\%},\hat{N}_{97,5\%}\right].
    \end{equation}
    \end{itemize}

\end{enumerate}

% To odwołanie nie wiem dlaczego nie działa - ścina nazwę autora, którą jest organizacja. Dlatego wpisałem odwołanie ręcznie \citep{international1995theory}.

%%%%%%%%%%%%%%%%%%%%%%%%%%%%%%%%%%%%%%%%%%%%%%%%%%%%%%%%%%%%%%%%%%%%%%%%%%%%%%%%%%%%%%%%%%%%%

\begin{center}
\DIFdelbegin \section*{\DIFdel{ŹRÓDŁA DANYCH}}
%DIFAUXCMD
\DIFdelend \DIFaddbegin \DIFadd{ŹRÓDŁA DANYCH
}\DIFaddend \end{center}

\begin{center}
\DIFdelbegin \subsection*{\DIFdel{Wybór zbiorów}}
%DIFAUXCMD
\DIFdelend \DIFaddbegin \textbf{\DIFadd{Wybór zbiorów}}
\DIFaddend \end{center}

Postawione w~artykule cele realizowane były z~wykorzystaniem informacji pochodzących z~zasobów informacyjnych statystyki publicznej za lata 2015 i~2016 (PBSSP), w~szczególności z~wykorzystaniem danych administracyjnych i~statystycznych gromadzonych w~ramach badań: „Zasoby migracyjne w~Polsce”, „Cudzoziemcy w~Polsce. Legalizacja pobytu cudzoziemców na terytorium RP”, „Operat do Badań Społecznych”, „Charakterystyka demograficzno-społeczna i~ekonomiczna gospodarstw domowych i~rodzin”, „Badanie Aktywności Ekonomicznej Ludności”.

Po dokonaniu analizy i~oceny zasobów danych jako główne źródła administracyjne wykorzystywane w~modelach log-liniowych wykorzystano:

\begin{itemize}
    \item System „Pobyt” (Urząd do Spraw Cudzoziemców -- UdSC) -- zbiór rejestrów, ewidencji i~wykazu w~sprawach cudzoziemców w~zakresie wydanych zezwoleń na pobyt,
    %\item Ministerstwa Rodziny, Pracy i~Polityki Społecznej – w~zakresie zezwoleń na pracę i~oświadczeń pracodawców o powierzeniu pracy cudzoziemcom,
    \item Rejestr PESEL (Ministerstwo Cyfryzacji) -- w~zakresie cudzoziemców zameldowanych wyłącznie na pobyt stały,
    \item Centralny Rejestr Ubezpieczonych (Zakład Ubezpieczeń Społecznych -- ZUS) -- w~zakresie ubezpieczonych cudzoziemców oraz członków ich rodzin (udostępniony zbiór niestety nie pokrywał wszystkich ubezpieczonych).
    %\item Centralny Rejestr Podmiotów – Krajowa Ewidencja Podatników (CRP KEP) w~zakresie danych o osobach fizycznych – cudzoziemcach.
\end{itemize}

W projekcie, który jest podstawą tego artykułu, wykorzystano więcej źródeł, które z~racji innego zastosowania oraz ograniczonego miejsca nie zostały tutaj uwzględnione. 

\begin{center}
\DIFdelbegin \subsection*{\DIFdel{Przygotowanie zbiorów danych na potrzeby badania}}
%DIFAUXCMD
\DIFdelend \DIFaddbegin \textbf{\DIFadd{Przygotowanie zbiorów danych na potrzeby badania}}
\DIFaddend \end{center}

Wejściowe zbiory wykorzystane w~projekcie badawczym poddano przetwarzaniu umożliwiającemu porównywanie, łączenie i~analizę danych z~różnych źródeł oraz oszacowanie wyników. W tym obszarze prac można wyróżnić kilka, wzajemnie przenikających i~uzupełniających się, grup działań.

\textbf{Dobór podmiotowy i~przedmiotowy} -- w~tej fazie prac -- na podstawie wstępnej analizy  zawartości zbiorów wejściowych oraz stosownie do przyjętego zakresu przedmiotowego badania i~przesłanek metodologicznych –- dokonano selekcji potencjalnie przydatnych zmiennych ze zbiorów. Z kolei, stosownie do zakresu podmiotowego badania, zastosowano dobory rekordów, tzn. tak, aby dotyczyły one cudzoziemców (np. w~przypadku zbiorów z~badań obejmujących szersze kategorie ludności) w~odpowiednich do założonych w~badaniu momentów obserwacji (31.12.2015 r. i~31.12.2016 r.) pod względem okresu przebywania w~Polsce (w przypadku zbiorów rejestrowych odnotowujących fakty i~daty dotyczące pobytu).

\textbf{Wyliczanie cech pochodnych na podstawie przekształceń surowych danych} -- w ramach tej grupy działań wykonano szereg wyliczeń i~przekształceń surowych danych, mających przede wszystkim na celu: (1) utworzenie (wyprowadzenie) cech potrzebnych do opisu badanej populacji, czyli tego typu operacje, jak np. wyliczanie okresu pobytu cudzoziemca na podstawie dat zarejestrowanych w~dokumentach; (2) zapewnienie zgodności definicyjnej i~zakresowej cech pochodzących z~różnych źródeł -- np. dostosowanie różnorakich konwencji zapisów kraju obywatelstwa do ujednoliconego słownika kodów krajów.

\textbf{Redukcja nadmiarowych danych}  -- ta faza składała się z~dwóch kroków: (1) \textit{deduplikacji w~obrębie pojedynczych zbiorów danych}, polegającej na wykrywaniu i~usuwaniu zwykłych (ewidentnych) dubli – powielonych rekordów danych, czyli takich, które pomimo różnych technicznych (bazodanowych) identyfikatorów rekordu, zawierały dokładne powtórzenie wszystkich wartości; (2) \textit{niwelowaniu redundancji podmiotowej danych w~kilku zbiorach jednego rejestru wartości/danych}. Łączenie (parowanie) poszczególnych zbiorów w~ramach danego rejestru i~wykrywanie rekordów dotyczących tych samych podmiotów (osób), a~następnie –- w~oparciu o przesłanki merytoryczne i~utworzone na ich podstawie hierarchie adekwatności -- dokonano wyboru najbardziej odpowiedniego rekordu reprezentującego danego cudzoziemca w~rejestrze. W konsekwencji, w~odniesieniu do określonego rejestru powstawał -- w~zależności od potrzeb -- jeden zbiór zawierający dane dotyczące unikalnych jednostek lub kilka zbiorów, ale podmiotowo rozłącznych.

\textbf{Wyodrębnianie podstawowych jednostek/podmiotów badania} -- w~wielu zbiorach rejestrowych dedykowanych cudzoziemcom podstawowe jednostki danych (rekordy) nie odnoszą się bezpośrednio do pojedynczych osób, lecz do różnego rodzaju faktów dotyczących osób. Stąd niezbędne były przekształcenia i~transformacje zbiorów wejściowych, w~wyniku których otrzymywano rekordy danych odnoszące się do osób. W~szczególności były to działania oparte na: (1) 	grupowaniu (agregowaniu) rekordów danych, w~ramach którego utworzono rekordy dla osób oraz wyprowadzano za pomocą operacji i~funkcji agregujących przewidziane w~badaniu cechy charakteryzujące cudzoziemców lub cechy pomocnicze; (2)	restrukturyzacji (transpozycji) danych, w~wyniku których pewne różnorodne wartości dotyczące jednej osoby zarejestrowane w~kilku rekordach (różne warianty cechy) zapisywano w~kolumnach jednego rekordu odnoszącego się do osoby.

\textbf{Łączenie (parowanie) rekordów z~różnych zbiorów} -- operacje łączenia przeprowadzane były zarówno w~obrębie zbiorów pochodzących z~jednego rejestru -- zazwyczaj na podstawie przygotowanego przez gestora sztucznego identyfikatora rekordów/osób -- jak i~kojarzenia zbiorów z~różnych rejestrów czy badań, w~tym wypadku -- na ogół za pomocą uniwersalnego identyfikatora (numer PESEL) lub na podstawie kombinacji wartości kilku cech\DIFaddbegin \footnote{\DIFadd{W~łączeniu zbiorów wykorzystanych w~niniejszym artykule zastosowano również parowanie według kluczy alternatywnych wobec numeru PESEL -- głównie w~odniesieniu do łączenia zbioru UDSC, w~którym znaczna część rekordów nie miała numeru PESEL. Wykorzystano w~nich, jako klucza podstawowego, zestawienia uwzględniającego datę urodzenia, płeć i~kraj obywatelstwa oraz -- w~zależności od rodzaju podejścia i~dostępności zapisów w~kolumnach -- różne kombinacje spośród takich cech jak: kod gminy, nazwa miejscowości i~numer budynku. W~tym wypadku liczba połączeń niejednoznacznych była stosunkowo niewielka i~ostatecznie zrezygnowano z~łączenia stochastycznego.}}\DIFaddend .


%%%%%%%%%%%%%%%%%%%%%%%%%%%%%%%%%%%%%%%%%%%%%%%%%%%%%%%%%%%%%%%%%%%%%%%%%%%%%%%%%%%%%%%%%%%%%
\begin{center}
\DIFdelbegin \section*{\DIFdel{WYNIKI BADANIA}}
%DIFAUXCMD
\DIFdelend \DIFaddbegin \DIFadd{WYNIKI BADANIA
}\DIFaddend \end{center}

\begin{center}
\DIFdelbegin \subsection*{\DIFdel{Spełnianie założeń metody capture-recapture}}    
%DIFAUXCMD
\DIFdelend \DIFaddbegin \textbf{\DIFadd{Spełnianie założeń metody capture-recapture}}    
\DIFaddend \end{center}


W związku z~wykorzystaniem w~niniejszym artykule metody \textit{capture-recapture} bazującej na wielu źródłach, do oszacowania liczby cudzoziemców poza dostępnymi statystycznymi źródłami danych, należy w~pierwszej kolejności określić przyjęte założenia metodologiczne. Poniżej przedstawiono listę kluczowych założeń, których spełnienie jest niezbędne z~punktu widzenia przyjętych rozwiązań modelowych. Wskazano również działania, które miały na celu ich spełnienie. 

\textbf{Definicje populacji we wszystkich rozważanych źródłach są takie same} -- określono populację cudzoziemców jako osoby posiadające obywatelstwo inne niż polskie w~wieku 18+, które przebywały w~Polsce w~końcu 2015 i~2016 roku. Każde z~wykorzystanych źródeł zostało ograniczone do tej populacji. 

\textbf{Populacja jest zamknięta} -- zakłada się, że w~badanym okresie wielkość populacji jest stała. Ponadto należy podkreślić, że wszystkie rejestry były aktualne na ten sam dzień, tj. 31.12.2016 r., dlatego podjęto następujące kroki przy wyodrębnianiu populacji na lata 2015 i~2016:

\begin{itemize}
    \item populacja na dzień 31.12.2015 r.:
    \begin{itemize}
        \item na podstawie PESEL, UdSC i~ZUS wybrano tylko osoby urodzone przed 31 grudnia 1997 r.,
        \item na podstawie UdSC wybrano tylko te osoby, które miały decyzję umożliwiającą pobyt w~Polsce wydaną między 01.01.2015 r. oraz 31.12.2015 r.
    \end{itemize}
    \item populacja na dzień 31.12.2016 r.:    
    \begin{itemize}
        \item na podstawie PESEL, UdSC i~ZUS wybrano tylko osoby urodzone przed 31 grudnia 1998 r.,
        \item na podstawie UdSC wybrano tylko tych, którzy mieli decyzję umożliwiającą pobyt w~Polsce wydaną między 01.01.2016 r. oraz 31.12.2016 r.
    \end{itemize}
\end{itemize}

W przypadku rejestru UdSC nie wyłączono z~analizy cudzoziemców, którym upłynęła data ważności dokumentu wydanego przez Urząd do Spraw Cudzoziemców w~ciągu 2016 r., tj. przed 31.12.2016 r. ponieważ mogły przebywać w~Polsce nielegalnie. 

\textbf{Źródła danych są niezależne} -- w przypadku źródeł administracyjnych systemy powinny być niezależne (\DIFdelbegin \DIFdel{na przykład }\DIFdelend w~sensie \DIFdelbegin \DIFdel{prawnym}\DIFdelend \DIFaddbegin \DIFadd{statystycznym}\DIFaddend ), aby możliwe było zastosowanie metody \textit{capture-recapture} wykorzystującej modele log-liniowe. \DIFaddbegin \DIFadd{Niezależność w~kontekście źródeł administracyjnych oznacza, że prawdopodobieństwo znalezienia się jednostki w~jednym źródle nie zależy od przynależności tej jednostki do drugiego źródła. }\DIFaddend Ostatecznie na potrzeby artykułu wykorzystano kombinacje trzech źródeł danych, które umożliwiają spełnienie tego założenia (tj. PESEL, UdSC i~ZUS). Głównym uzasadnieniem wyboru tych źródeł danych było ich bieżące wykorzystywanie w~statystyce publicznej na potrzeby innych badań (nie wymagało to pozyskania danych spoza PBSSP) oraz pokrycie tej samej populacji. 

\textbf{Brak błędów nadreprezentacji i~duplikatów} -- zakłada się, że źródła są pozbawione błędów nadreprezentacji, tj. źródła zawierają wyłącznie jednostki z~badanej populacji oraz zostały zdeduplikowane. Podstawowym źródłem był zintegrowany zbiór danych powstały na podstawie łączenia kilku rejestrów administracyjnych i~zdeduplikowany. Rejestr ten zawierał zmienną dotyczącą jakości danego rekordu, który jest przybliżeniem błędu nadreprezentacji. Dla rekordów występujących w~PESEL, ZUS lub UDSC wyodrębniono te, dla których określono kod jakości 1 oznaczający \textit{sytuacja referencyjna (istnienie osoby potwierdzone)}, 3 wskazujący \textit{osoby w~wieku 90+} oraz kod 6 oznaczający \textit{osobę zidentyfikowaną tylko w~jednym rejestrze}, który był wyznaczony przed dołączeniem rejestru UDSC. Dodatkowo, przyjęto przy tym założenie, że cudzoziemcy będący w~rejestrach przebywają na terenie Polski niezależnie od tego czy mają ustalone miejsce pobytu. Jest to kluczowe zwłaszcza w~przypadku systemu „Pobyt”, którego gestorem jest UdSC. 

\textbf{Każdą jednostkę będzie można zidentyfikować i~połączyć między źródłami bez błędów} -- w~tym celu zastosowano  integrację danych za pomocą identyfikatora PESEL lub kombinacji zmiennych jednoznacznie wskazujących daną osobę (łączenie deterministyczne). Nie dokonywano łączenia probabilistycznego. 

\textbf{Prawdopodobieństwa włączenia do co najmniej jednego z~rejestrów powinny być jednorodne} -- aby spełnić to założenie w~procesie \DIFdelbegin \DIFdel{modelowania }\DIFdelend \DIFaddbegin \DIFadd{estymacji }\DIFaddend wykorzystano modele zawierające następujące zmienne: 1) kraj obywatelstwa, 2) płeć, 3) wiek (2 grupy) i~4) województwo (16 oraz nieustalone). Wybór \DIFaddbegin \DIFadd{zmiennych }\DIFaddend podyktowany był \DIFdelbegin \DIFdel{zapisami projektu, podstawą którego jest niniejszy artykuł. }\DIFdelend \DIFaddbegin \DIFadd{z~jednej strony ich dostępnością, z~drugiej zaś koniecznością spełnienia warunku, aby w~odpowiednio utworzonych grupach prawdopodobieństwa włączenia cudzoziemca do danego źródła były jednakowe. Jest to jeden ze sposobów spełnienia założenia dotyczącego homogeniczności prawdopodobieństw, który rekomenduje się w literaturze poświęconej metodom capture-recapture \mbox{%DIFAUXCMD
\citep[por. ][s. 2]{van2012people}}\hspace{0pt}%DIFAUXCMD
.
}\DIFaddend 

\begin{center}
\DIFdelbegin \subsection*{\DIFdel{Opis danych}}    
%DIFAUXCMD
\DIFdelend \DIFaddbegin \textbf{\DIFadd{Opis danych}}    
\DIFaddend \end{center}

W\DIFdelbegin \DIFdel{Tabeli }\DIFdelend \DIFaddbegin \DIFadd{~Tablicy }\DIFaddend \ref{tabela-dane} przedstawiono liczbę cudzoziemców według występowania w~trzech źródłach według badanych lat. 

\begin{table}[ht]
\centering
\caption{Zestawienie liczby cudzoziemców w~wieku 18+ według PESEL, UdSC i~ZUS zgodnie ze stanem na 31.12.2015 r. i~31.12.2016.r}
\label{tabela-dane}
\begin{tabular}{llll|rr|r}
  \hline
  \multicolumn{7}{c}{\textbf{Stan na 31.12.2015 r.}} \\
  \hline
  &             &&        &  \multicolumn{2}{c|}{UdSC} & \\ 
  &             &&         &  Nie & Tak & $\sum$ \\ 
    \hline
  PESEL & Nie   &ZUS& Nie &               -- &  30 090 & 30 090 \\ 
        &       && Tak  &            3 821 &   5 583 &  9 404 \\ 
  \hline        
        & Tak   &ZUS& Nie &        7 042 &   7 476 & 14 518 \\ 
        &       && Tak &           4 620 &   9 871 & 14 491 \\ 
 \hline
  $\sum$ & &  &&           15 483 &  53 020 & 68 503 \\ 
  \hline
  \multicolumn{7}{c}{\textbf{Stan na 31.12.2016 r.}}\\
  \hline
  & &           &&  \multicolumn{2}{c|}{UdSC} & \\ 
  &             &&         &  Nie & Tak & $\sum$ \\ 
    \hline
  PESEL & Nie   &ZUS& Nie &             -- &  92 106 &  92 106 \\ 
  &             && Tak  &           3 821 &  11 224 &  15 045 \\ 
  \hline
  & Tak         &ZUS& Nie &           7 115 &  16 549 &  23 664 \\ 
  &             && Tak  &           4 641 &  18 951 &  23 592 \\
    \hline
   $\sum$ & &    &&          15 577 & 138 830 & 154 407 \\ 
   \hline
\DIFdelbeginFL %DIFDELCMD < 

%DIFDELCMD < %%%
\DIFdelendFL \end{tabular}
  \DIFaddbeginFL \begin{flushleft}
\small{
Źródło: Opracowanie własne na podstawie podstawie danych PESEL, ZUS i UdSC.
}
\end{flushleft}
\DIFaddendFL \end{table}

Wartość ,,Tak’’ oznacza, że cudzoziemiec został zidentyfikowany, a~wartość ,,Nie” oznacza, że nie został zidentyfikowany w~danym źródle. W~przypadku kombinacji PESEL, UdSC, ZUS w~2015 roku wykorzystano informacje o ponad 68,5 tys., a~w~2016 roku dla blisko 154 tys. cudzoziemców. 
\DIFdelbegin \DIFdel{W przypadku kombinacji PESEL, UdSC i~KEP w~2015 roku wykorzystano dane o blisko 69 tys., a~w~2016 o 154,5 tys. cudzoziemców. 
}\DIFdelend 

W odniesieniu do 2015 roku jedynie 9 871 cudzoziemców było zidentyfikowanych jednocześnie w~PESEL, UdSC i~ZUS. W 2016 roku 18 951 cudzoziemców występowało jednocześnie w~PESEL, UdSC i~ZUS. W tabelach pojawia się również wartość “--”, która oznacza nieznaną liczbę cudzoziemców będących poza wymienionymi rejestrami. 

Głównym celem niniejszego artykułu jest oszacowanie liczby cudzoziemców będących poza tymi rejestrami, tj. nieznanej wartości liczbowej na przecięciu pól: PESEL = Nie, UdSC = Nie i~ZUS = Nie. Na potrzeby estymacji tej liczebności wykorzystano wspomniane już modele log-liniowe.

\begin{center}
\subsection*{Dobór modelu}    
\end{center}

Zgodnie z~literaturą  poświęconą szacowaniu wielkości nieznanej populacji założono, że prawdopodobieństwo pokrycia przez określone źródła danych nie jest jednakowe. Dlatego na potrzeby procesu modelowania wykorzystano następujące zmienne:

\begin{itemize}
    \item Płeć -- 	1 = Mężczyzna, 2 = Kobieta,
    \item Wiek -- Produkcyjny (18--59 dla kobiet, 18--64 dla mężczyzn), Poprodukcyjny – 60+ dla kobiet, 65+ dla mężczyzn.
    \item Kraj obywatelstwa -- UE, Armenia, Mołdawia, Białoruś, Rosja, Ukraina, Wietnam, Pozostałe.
    \item Województwo: 16 województw kodowanych 1,...16, nieustalone (jeżeli nie zostało określone miejsce pobytu).
\end{itemize}

Na potrzeby wyboru końcowego modelu log-liniowego, który wykorzystano w~procesie estymacji liczby cudzoziemców w~Polsce w~odpowiednich przekrojach, w~pierwszej kolejności dokonano zakodowania zmiennych, zgodnie z~symbolicznym zapisem (notacja nawiasowa) charakterystycznym dla modeli log-liniowych. Poniższy opis dotyczy wyników uzyskanych dla modelu wykorzystującego województwa. Przyjęto zatem następujące oznaczenia:  PESEL = P, UdSC = U, ZUS  = Z, Płeć = S (od angielskiego słowa sex), Wiek = A (od angielskiego słowa age), Kraj obywatelstwa = C (od angielskiego słowa citizenship), Województwo = V (od angielskiego słowa voivodeship).

Procedurę modelowania przeprowadzono oddzielnie dla lat 2015 i~2016 oraz dla kombinacji źródeł. Oznacza to, że ostatecznie przeprowadzono \DIFdelbegin \DIFdel{cztery }\DIFdelend \DIFaddbegin \DIFadd{dwie }\DIFaddend niezależne procedury szacunku wielkości populacji. 

\DIFdelbegin \DIFdel{Nie rozważano modelu z~trzema interakcjami, ponieważ liczba możliwych interakcji byłaby bardzo duża, co skomplikowałoby model i~mogłoby doprowadzić do jego przeuczenia przy relatywnie niewielkiej liczbie obserwacji. 
}%DIFDELCMD < 

%DIFDELCMD < %%%
\DIFdel{Tabela }\DIFdelend \DIFaddbegin \DIFadd{Tablica }\DIFaddend \ref{jakosc} zawiera zestawienie wybranych miar jakości dla zastosowanych modeli log-liniowych według kombinacji źródeł oraz roku. W przypadku modelu opartego na kombinacji PESEL, UdSC i~ZUS model 2 i~2s okazał się identyczny w~2015 roku (co potwierdzają kryteria informacyjne), a~w~przypadku 2016 roku model 2s był nieznacznie lepszy od modelu 2, ponieważ zarówno kryteria informacyjne AIC, jak i~BIC, są niższe. 

\begin{table}[ht!]
    \centering
    \caption{Wybrane miary jakości modeli log-liniowych według roku}
    \label{jakosc}
    \begin{tabular}{rrrrrrrrr}
\hline
Rok & Model & Dewiancja M0 & df M0 & G2 & AIC & BIC & Dewiancja & df r \\ 
\hline
2015 & 1 & 216 672 & 2 218 & -24 972 & 50 003 & 50 168 & 41 580 & 2 190 \\ 
 & 2 & 216 672 & 2 218 & -8 909 & 18 349 & 19 860 & 9 453 & 1 954 \\ 
 & 2s & 216 672 & 2 218 & -8 909 & 18 349 & 19 860 & 9 453 & 1 954 \\ 
2016 & 1 & 651 403 & 2 399 & -50 619 & 101 295 & 101 463 & 9 1543 & 2 371 \\ 
& 2 & 651 403 & 2 399 & -12 260 & 25 051 & 26 583 & 14 827 & 2 135 \\ 
& 2s & 651 403 & 2 399 & -12 260 & 25 049 & 26 575 & 14 827 & 2 136 \\ 
\hline
    \end{tabular}
\DIFdelbeginFL %DIFDELCMD < \legend{Źródło: opracowanie własne. Wyjaśnienia: 1 = model wyłącznie z~efektami głównymi, 2 = model z~efektami głównymi i~interakcjami pierwszego rzędu, 2s = model 2 z~zastosowaną procedurą krokową (s pochodzi od step, które odnosi się do pojęcia regresji krokowej, ang. \textit{stepwise selection; stepwise regression}), df = stopnie swobody, M0 –-model jedynie z~wyrazem wolnym (inaczej model pusty), df r –- różnica między liczbą stopni swobody modelu pustego, a~modelu w~danym wierszu.}
%DIFDELCMD < %%%
\DIFdelendFL \DIFaddbeginFL \legend{Źródło: opracowanie własne na podstawie danych PESEL, ZUS i~UdSC. Wyjaśnienia: 1 = model wyłącznie z~efektami głównymi, 2 = model z~efektami głównymi i~interakcjami pierwszego rzędu, 2s = model 2 z~zastosowaną procedurą krokową (s pochodzi od step, które odnosi się do pojęcia regresji krokowej, ang. \textit{stepwise selection; stepwise regression}), df = stopnie swobody, M0 –-model jedynie z~wyrazem wolnym (inaczej model pusty), df r –- różnica między liczbą stopni swobody modelu pustego, a~modelu w~danym wierszu.}
\DIFaddendFL \end{table}

\begin{center}
\DIFdelbegin \subsection*{\DIFdel{Estymacja punktowa i~przedziałowa}}    
%DIFAUXCMD
\DIFdelend \DIFaddbegin \textbf{\DIFadd{Estymacja punktowa i~przedziałowa}}    
\DIFaddend \end{center}

\DIFdelbegin \DIFdel{Tabela }\DIFdelend \DIFaddbegin \DIFadd{Tablica }\DIFaddend \ref{tab-modele-zapis} przedstawia postać finalnego modelu wraz z~oszacowaną wielkością populacji cudzoziemców w~Polsce w~latach 2015 i~2016 oraz 95\% bootstrapowym przedziałem ufności. Model dla 2015 r. nieznacznie różni się od modelu dla 2016 r., ponieważ posiada jeden dodatkowy element –- interakcję między płcią, a~wiekiem (oznaczone w~tabeli jako [SA]). Wynik modelowania sugeruje, że prawdopodobieństwa pokrycia przez badane źródła danych jest stałe, co skutkuje stabilnością modelu w~czasie. 

\begin{table}[ht!]
    \centering
    \caption{Postać ostatecznego modelu log-liniowego (notacja nawiasowa) wraz z~oszacowaną wielkością populacji cudzoziemców w~Polsce, 95\% przedziałem ufności i~precyzją oszacowań (w \%)}
    \label{tab-modele-zapis}
    \begin{tabular}{rp{7cm}rrr}
    \hline
    Rok & Model & $\hat{N}$ & Przedział ufności & Precyzja \\
    \hline
      2015  & [P][Z][U][V][S][A][C] [PZ][PU][PV][PS][PA][PC][ZU][ZV] [ZA][ZC][UV][UC][VS][VA][VC] [SA][AC][UA][US][ZS][\textit{SC}] &  507 693 & 
      (369 135, 724 407) & 
      17,64 \\

      2016   & [P][Z][U][V][S][A][C] [PZ][PU][PV][PS][PA][PC][ZU][ZV] [ZA][ZC][UV][UC][VS][VA][VC] [SA][AC][UA][US][SC]  &  743 665 & 
      (600 796, 943 124) & 
      11,70 \\
\hline
    \end{tabular}
    \DIFdelbeginFL %DIFDELCMD < 

%DIFDELCMD < %%%
\DIFdelendFL \DIFaddbeginFL \legend{Źródło: opracowanie własne na podstawie danych PESEL, ZUS i~UdSC. Wyjaśnienia: Notacja nawiasowa oznacza efekty główne oraz interakcje. Litery oznaczają odpowiednio PESEL = P, UdSC = U, ZUS  = Z, Płeć = S, Wiek = A, Kraj obywatelstwa = C, Województwo = V.}
\DIFaddendFL \end{table}

Według szacunków liczba cudzoziemców w~wieku 18 lat i~więcej przebywających w~Polsce w~końcu 2015 r. wynosiła 507,7 tys. (95\% przedział ufności od 369,1 tys. do 724,4 tys.). Liczba ta – oprócz cudzoziemców zameldowanych na pobyt czasowy –- obejmowała również cudzoziemców zameldowanych na pobyt stały (takich osób według rejestru PESEL było 39,1 tys.). Dla porównania, zgodnie z~publikacjami ZUS, liczba ubezpieczonych cudzoziemców zgłoszonych do ubezpieczeń emerytalnych i~rentowych wynosiła 184\DIFaddbegin \DIFadd{~}\DIFaddend 188 w~końcu 2015 roku. 

Analogiczne oszacowano liczbę cudzoziemców w~wieku 18 lat i~więcej przebywających w~Polsce w~końcu 2016 r. na 743,7 tys. (95\% poziom ufności 600,8--943,1 tys). Liczba ta – oprócz cudzoziemców zameldowanych na pobyt czasowy – obejmowała również cudzoziemców zameldowanych na pobyt stały (takich osób wg rejestru PESEL było 41,4 tys.).  W 2016 r. odnotowano wyraźny wzrost liczby cudzoziemców w~stosunku do roku poprzedniego. Wzrosła liczba obywateli Ukrainy, Białorusi, Rosji, Wietnamu i~innych krajów spoza UE –- liczba obywateli UE nieznacznie zmniejszyła się. Według statystyk ZUS, liczba \DIFdelbegin \DIFdel{cudzoziemców }\DIFdelend \DIFaddbegin \DIFadd{osób fizycznych o~obywatelstwie innym niż polskie }\DIFaddend zgłoszonych do ubezpieczeń społecznych i~rentowych w~końcu 2016 roku wynosiła 293\DIFdelbegin \DIFdel{188.
}%DIFDELCMD < 

%DIFDELCMD < %%%
%DIF < % do podsumowania
%DIF < %% UDSC -- karty pobytu 266 218 (2016), 211 869 (2015)
%DIF < %% MRPiPS -- zezwolenia 139 119 (2016), 74 149 (2015)
%DIF < %% MRPiPS -- oswiadczenia 1 314 127 (2016) -- (2015)
%DIF < %% ZUS -- 293 188 (2016), 184 188 (2015)
%DIFDELCMD < 

%DIFDELCMD < %%%
\DIFdel{Wśród }\DIFdelend \DIFaddbegin \DIFadd{~188, natomiast liczba }\DIFaddend cudzoziemców \DIFdelbegin \DIFdel{przebywających w~Polsce zdecydowanie przeważają obywatele krajów trzecich (co oznacza każdą osobę, która nie jest obywatelem Unii Europejskiej w~rozumieniu art. 17 ust. 1 Traktatu, w~tym bezpaństwowców). }\DIFdelend \DIFaddbegin \DIFadd{pracowników zgłoszonych do tego samego ubezpieczenia wynosiła    169~350. }\DIFaddend Tablica \ref{tab-szacunek-szczegoly} przedstawia szczegółowe zestawienie wyników w~podziale na kraj obywatelstwa. 
\DIFdelbegin \DIFdel{Wynika to z~faktu, że polski rynek pracy jest atrakcyjny dla cudzoziemców zza naszej wschodniej granicy, z~jednej strony z~powodu bliskości geograficznej, sieci migracyjnych, które pozwalają zminimalizować koszty pobytu przynajmniej w~pierwszych tygodniach, zdecydowanie wyższych zarobków niż w~krajach rodzimych, z~drugiej -- liberalizacji zasad dostępu obywateli do polskiego rynku pracy. Uregulowania prawne wprowadzające uproszczoną procedurę zezwalają na podejmowanie pracy (oświadczenia pracodawców o powierzeniu pracy cudzoziemcowi) przez obywateli sześciu krajów trzecich: Armenii, Białorusi, Gruzji, Mołdawii, Rosji i~Ukrainy. Spośród krajów trzecich to właśnie obywatele Ukrainy stanowią największą zbiorowość. Szacuje się, że w~2015 r. przebywało 283,7 tys. (95\% przedział ufności: 203,9 tys. – 415,7 tys.), a~w~2016 r. 455,0 tys. (95\% przedział ufności: 361,5 tys. – 584,7 tys.) obywateli tego kraju.
}\DIFdelend 


\begin{table}[ht!]
\caption{Szacunek wielkości populacji cudzoziemców w~Polsce według kraju obywatelstwa}
\label{tab-szacunek-szczegoly}
    \centering
    \begin{tabular}{llrrrr}
    \hline
    Rok & Kraj & $\hat{N}$ &  \multicolumn{2}{c}{95\% Przedział ufności} & Precyzja \\
    \hline
Armenia & 2015  & 3 168 & 2 263 & 4 505 & 18,33 \\ 
 & 2016  & 4 773 & 3 897 & 6 032 & 11,35 \\ 
Białoruś & 2015  & 19 868 & 14 429 & 27 951 & 17,38 \\ 
 & 2016  & 25 813 & 20 832 & 32 569 & 11,81 \\ 
Mołdawia & 2015  & 2 693 & 1 613 & 4 227 & 25,59 \\ 
 & 2016  & 7 580 & 5 355 & 10 617 & 17,99 \\ 
Rosja & 2015  & 22 611 & 16 040 & 32 237 & 18,62 \\ 
 & 2016  & 25 534 & 20 685 & 32 344 & 12,07 \\ 
Ukraina & 2015  & 283 714 & 203 946 & 415 732 & 18,55 \\ 
 & 2016  & 454 974 & 361 512 & 584 696 & 12,27 \\ 
Wietnam & 2015  & 7 408 & 5 554 & 9 942 & 15,45 \\ 
 & 2016  & 11 728 & 10 008 & 14 170 & 9,10 \\ 
kraje EU & 2015 & 70 901 & 53 579 & 97 126 & 15,63 \\ 
 & 2016 & 59 571 & 50 914 & 71 169 & 8,77 \\ 
pozostałe & 2015  & 97 329 & 70 037 & 138 339 & 17,86 \\ 
 & 2016  & 153 692 & 124 170 & 196 140 & 12,06 \\ 
    \hline     
    \end{tabular}
    \DIFaddbeginFL \begin{flushleft}
\small{
Źródło: Opracowanie własne na podstawie podstawie danych PESEL, ZUS i UdSC.
}
\end{flushleft}
\DIFaddendFL \end{table}

\DIFaddbegin \DIFadd{Wśród cudzoziemców przebywających w~Polsce zdecydowanie przeważają obywatele krajów trzecich (co oznacza każdą osobę, która nie jest obywatelem Unii Europejskiej w~rozumieniu art. 17 ust. 1 Traktatu, w~tym bezpaństwowców). Wynika to z~faktu, że polski rynek pracy jest atrakcyjny dla cudzoziemców zza naszej wschodniej granicy, z~jednej strony z~powodu bliskości geograficznej, sieci migracyjnych, które pozwalają zminimalizować koszty pobytu przynajmniej w~pierwszych tygodniach, zdecydowanie wyższych zarobków niż w~krajach rodzimych, z~drugiej -- liberalizacji zasad dostępu obywateli do polskiego rynku pracy. Uregulowania prawne wprowadzające uproszczoną procedurę zezwalają na podejmowanie pracy (oświadczenia pracodawców o powierzeniu pracy cudzoziemcowi) przez obywateli sześciu krajów trzecich: Armenii, Białorusi, Gruzji, Mołdawii, Rosji i~Ukrainy. Spośród krajów trzecich to właśnie obywatele Ukrainy stanowią największą zbiorowość. Szacuje się, że w~2015 r. przebywało 283,7 tys. (95\% przedział ufności: 203,9 tys. - 415,7 tys.), a~w~2016 r. 455,0 tys. (95\% przedział ufności: 361,5 tys. - 584,7 tys.) obywateli tego kraju.
}\DIFaddend %%%%%%%%%%%%%%%%%%%%%%%%%%%%%%%%%%%%%%%%%%%%%%%%%%%%%%%%%%%%%%%%%%%%%%%%%%%%%%%%%%%%%%%%%%%%%
\DIFdelbegin %DIFDELCMD < \newpage
%DIFDELCMD < %%%
\DIFdelend \DIFaddbegin 

\DIFaddend \begin{center}
\DIFdelbegin \section*{\DIFdel{PODSUMOWANIE}} 
%DIFAUXCMD
\DIFdelend \DIFaddbegin \DIFadd{PODSUMOWANIE
}\DIFaddend \end{center}

 %DIF > % dopisać dyskusję wyników
Wybrana do oszacowania \DIFdelbegin \DIFdel{zasobu }\DIFdelend \DIFaddbegin \DIFadd{liczby }\DIFaddend cudzoziemców na krajowym rynku pracy metoda capture-recapture bazująca na modelach log-liniowych jak dotąd nie była stosowana w~badaniach statystycznych w~Polsce. Jedynymi doświadczeniami, z~których można było skorzystać, są empiryczne badania zrealizowane przez holenderskich i~norweskich badaczy.

Przedstawione w~artykule wyniki estymacji wielkości \DIFdelbegin \DIFdel{zasobów imigracyjnych}\DIFdelend \DIFaddbegin \DIFadd{populacji cudzoziemców}\DIFaddend , będące pochodną rezultatów otrzymanych we wspomnianym projekcie badawczym, mogą stanowić dobrą podstawę do wyprowadzenia (wtórnie) różnego rodzaju wskaźników dla wyodrębnionych jednostek terytorialnych, takich jak np. bilans migracyjny netto czy wskaźnik aktywności zawodowej cudzoziemców. Te ostatnie zaś mogą być wykorzystywane przez władze samorządowe, m.in. do monitorowania wielkości zatrudnienia, wysokości stopy bezrobocia czy popytu na pracę cudzoziemców o wysokich kwalifikacjach oraz oceny wpływu powierzania pracy cudzoziemcom na wysokość płac pracowników rodzimych. Dodatkowo przedstawione w~artykule szacunki mogą być wykorzystywane do monitorowania grup narażonych na wykluczenie zawodowe i~społeczne poprzez zapobieganie substytucji rodzimych zasobów pracy przez cudzoziemców. Wreszcie dane te będą mogły stanowić podstawę do prowadzenia analiz statystycznych na temat sytuacji społeczno-gospodarczej poszczególnych regionów kraju oraz prognoz ich rozwoju.

Przedstawione wyniki mają innowacyjny charakter ze względu na zastosowaną metodę opracowania szacunku oraz wykorzystane źródła danych. Należy jednak zaznaczyć, że w~trakcie badań napotkano poważne trudności. Najważniejszą ich przyczyną był fakt, że metoda szacunku oparta była na źródłach danych administracyjnych, pozyskiwanych w~ramach PBSSP dla innych badań, zatem o\DIFaddbegin \DIFadd{~}\DIFaddend zakresie informacyjnych zdefiniowanym przez określoną jednostkę/departament realizującą własne badanie. Jako przykład można wskazać rejestry ZUS pozyskiwane na potrzeby realizacji badań z~zakresu rynku pracy, które nie zawierały wszystkich ubezpieczonych cudzoziemców. Bardzo cenne zbiory dotyczące zezwoleń na pracę i~oświadczeń pracodawców o\DIFaddbegin \DIFadd{~}\DIFaddend zamiarze powierzenia pracy cudzoziemcowi nie zawierały cech identyfikacyjnych, w~związku z~czym nie można było ich połączyć deterministycznie z~innymi zbiorami. Zbiór PESEL z~kolei nie obejmował cudzoziemców przebywających czasowo i~zameldowanych w~gminach, którzy nie posiadali numeru PESEL. Co więcej, wszystkie wykorzystane rejestry były aktualne na dzień 31.12.2016 r., co mogło wpłynąć na wyniki uzyskane na dzień 31.12.2015 r. Tym samym wtórne wykorzystanie rejestrów i~zawartych w~nich zmiennych miało istotny wpływ zarówno na sam wybór źródeł danych oraz metodę, jak i~w konsekwencji na konstrukcję wskaźników.

Niezbędne jest również podjęcie prac nad rozpoznaniem innych źródeł, które ze względu na swój charakter i~zakres mogą być niezwykle \DIFdelbegin \DIFdel{cennym źródłem }\DIFdelend \DIFaddbegin \DIFadd{cenne }\DIFaddend do weryfikacji charakterystyki cudzoziemców, np. rejestr policji dotyczący cudzoziemców podejrzanych o~popełnienie przestępstw, zbiory Państwowej Inspekcji Pracy w~zakresie kontroli legalności zatrudniania cudzoziemców, Komendy Głównej Straży Granicznej w~zakresie legalności pobytów lub zbiory Ministerstwa Spraw Zagranicznych w~zakresie wiz. W\DIFaddbegin \DIFadd{~}\DIFaddend tym celu konieczne będzie nawiązanie bądź zintensyfikowanie współpracy z~gestorami poszczególnych rejestrów i~baz danych. Jednocześnie należy podkreślić, że w~dalszych pracach planuje się wykorzystanie metod uwzględniających łączenie deterministyczne i~probabilistyczne oraz analizy wrażliwości na złamania założeń metody capture-recapture, jak i~stosowanych modeli.


\begin{center}
\DIFdelbegin \section*{\DIFdel{PODZIĘKOWANIA}} 
%DIFAUXCMD
\DIFdelend \DIFaddbegin \DIFadd{PODZIĘKOWANIA
}\DIFaddend \end{center}

Niniejszy artykuł został przygotowany na podstawie raportu podsumowującego pracę badawczą pt.\textit{Cudzoziemcy na krajowym rynku pracy w~ujęciu regionalnym} zrealizowaną w~ramach projektu \textit{Wsparcie systemu monitorowania polityki spójności w~perspektywie finansowej 2014--2020 oraz programowania i~monitorowania polityki spójności po 2020 roku} współfinansowanego przez Unię Europejską ze środków Programu Operacyjnego Pomoc Techniczna 2014--2020. 

Autorzy artykułu składają podziękowania wszystkim osobom, które przyczyniły się do powstania raportu. W~szczególności są to: kierownik projektu - dyrektor Departamentu Badań Demograficznych Dorota Szałtys oraz członkowie zespołu badawczego: Michał Adamski, Mariusz Chmielewski, Piotr Filip, Daniel Godlewski, Tomasz Józefowski, Paweł Kaczorowski, Zofia Kostrzewa, Jacek Kowalewski, Arleta Olbrot-Brzezińska, Artur Owczarkowski, Joanna Stańczak, Karina Stelmach oraz Anna Wysocka\footnote{Pełen raport wraz z~załącznikami dostępny jest na stronie internetowej \url{http://stat.gov.pl/statystyka-regionalna/statystyka-dla-polityki-spojnosci/statystyka-dla-polityki-spojnosci-2016-2018/badania/rynek-pracy-ubostwo-i-wykluczenie-spoleczne/}}.

\vspace{0.3cm}

\noindent \textbf{Maciej Beręsewicz, Marcin Szymkowiak --} \textit{Katedra Statystyki Uniwersytet Ekonomiczny w~Poznaniu; Ośrodek Statystyki Małych Obszarów Urzędu Statystycznego w~Poznaniu}\\
\textbf{Grzegorz Gudaszewski --} \textit{Departament Badań Demograficznych, Główny Urząd Statystyczny}
%%%%%%%%%%%%%%%%%%%%%%%%%%%%%%%%%%%%%%%%%%%%%%%%%%%%%%%%%%%%%%%%%%%%%%%%%%%%%%%%%%%%%%%%%%%%%
\newpage

\bibliographystyle{plapa}
\bibliography{bibliography} 

%%%%%%%%%%%%%%%%%%%%%%%%%%%%%%%%%%%%%%%%%%%%%%%%%%%%%%%%%%%%%%%%%%%%%%%%%%%%%%%%%%%%%%%%%%%%%
 \DIFdelbegin %DIFDELCMD < 

%DIFDELCMD < \newpage
%DIFDELCMD < 

%DIFDELCMD < %%%
\textbf{\DIFdel{Summmary.}} %DIFAUXCMD
\textit{\DIFdel{The main aim of this article is to present selected results obtained in the study entitled ,,Foreigners in the national labour market – a~regional approach'' implemented in cooperation with Statistics Poland, the Statistical Office in Poznań and the University of Economics and Business in Poznań. The article describes methods for estimating the number of foreigners in Poland, with particular emphasis on foreigners working in Poland, using data from administrative and statistical sources and techniques of the capture-recapture method based on log-linear models. This is the first comprehensive analysis of an attempt to estimate the number of foreigners in Poland, which is part of the broader current of research focusing on hard-to-survey populations.
}}
%DIFAUXCMD
%DIFDELCMD < 

%DIFDELCMD < %%%
\textbf{\DIFdel{Keywords:}} %DIFAUXCMD
\DIFdel{estimation the number of foreigners in Poland, hard-to-survey population,  capture-recapture methods, log-linear analysis, admministrative registers
 }\DIFdelend\end{document}
%%%%%%%%%%%%%%%%%%%%%%%%%%%%%%%%%%%%%%%%%%%%%%%%%%%%%%%%%%%%%%%%%%%%%%%%%%%%%%%%%%%%%%%%%%%%%